\section{SSD}
\begin{definitions}
$$C$$ & Lösch/Schreib-Zyklen pro Speicherzelle                                  \\
$$T_{L}$$ & Lebensdauer                                                         \\
$$B_{G}$$ & Gesamtanzahl der Blöcke                                             \\
$$B_{T}$$ & Anzahl der Blöcke welche pro Zeiteinheit geschrieben werden         \\
$$P_{D}$$ & Prozentanteil der dynamischen Daten                                 \\
$$P_{S}$$ & Prozentanteil der reservierten Blöcke für das Bad Block Management
\end{definitions}

\[
B_{G} = \cfrac{\text{Kapazität}}{\text{Blockgröße}} \cdot ( 1 - P_{S} )
\]

\important
Wenn es heißt: \emph{Eine typische Datei, die zu den Dynamischen Daten gerechnet wird,
belegt im Mittel $x$ Blöcke ...}, dann muss wie flogt vorgegangen werden:\par
$A_{DBF} = \text{Anzahl Dynamischener Blöcke pro Datei}$
\[
 \text{Anzahl}_\text{Datein} = \cfrac{\text{Anzahl}_\text{Dynamischener Blöcke}}{\text{Anzahl}_\text{Dynamischener Blöcke pro Datei}} 
\]
\[
 B_{T} = \text{Anzahl}_\text{Dynamischener Blöcke  pro Datei} \cdot \text{Anzahl}_\text{Datein} \cdot \text{Zeiteinheit}
\]

\subsection*{Static Wear Levelling}
\[
T_{L} = \cfrac{C \cdot B_{G}}{B_{T}}
\]

\subsection*{Dynamic Wear Levelling}
\[
T_{L} = \cfrac{C \cdot ( B_{G} \cdot P_{D} )}{B_{T}}
\]

