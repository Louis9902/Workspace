\section{Multithreading}
Beispiel Verteilung von 3 Threads auf 4 Ausfüehrungseinheiten, dabei benötigt die Ausführung
einer Instruktion genau eine Taktzyklus:
\subsubsection*{Thread X}
\begin{tabular}{|c|llllll|}\hline
  X  & 1 & 2 & 3 & 4 & 5 & 6\\\hline
AE 1 & X1 & X4 & × & × & × & X6\\
AE 2 & X2 & X5 & × & × & × & X7\\
AE 3 & X3 &    & × & × & × &   \\
AE 4 &    &    & × & × & × &   \\\hline
\end{tabular}\par
\subsubsection*{Thread Y}
\begin{tabular}{|c|llllll|}\hline
  Y  & 1 & 2 & 3 & 4 & 5 & 6\\\hline
AE 1 & Y1 & Y2 & Y4 & × & × & X7\\
AE 2 &    & Y3 & Y5 & × & × & X8\\
AE 3 &    &    & Y6 & × & × &   \\
AE 4 &    &    &    & × & × &   \\\hline
\end{tabular}\par
\subsubsection*{Thread Z}
\begin{tabular}{|c|llllll|}\hline
  Z  & 1 & 2 & 3 & 4 & 5 & 6\\\hline
AE 1 & Z1 & × & Z5 & × & Z7 & ×\\
AE 2 & Z2 & × & Z6 & × &    & ×\\
AE 3 & Z3 & × &    & × &    & ×\\
AE 4 & Z4 & × &    & × &    & ×\\\hline
\end{tabular}\par

\paragraph{Feinkörniges Multithreading} Mit Round-Robin Ausführungsreihenfolge wird in jedem
Taktzyklus ein anderer Thread ausgeführt. Falls ein Thread keine Aufgaben hat, wird nicht
zurück gegangen.\par
\begingroup\setlength\tabcolsep{2pt}
\begin{tabular}{|c|llllllllll|}\hline
  ×  & 1 & 2 & 3 & 4 & 5 & 6 & 7 & 8 & 9 & 10\\\hline
AE 1 & X1 & Y1 & Z1 & X4 & Y2 & Z5 & Y4 & Z7 & X6 & Y7\\
AE 2 & X2 &    & Z2 & X5 & Y3 & Z6 & Y5 &    & X7 & Y8\\
AE 3 & X3 &    & Z3 &    &    &    & Y6 &    &    &   \\
AE 4 &    &    & Z4 &    &    &    &    &    &    &   \\\hline
\end{tabular}
\endgroup

\paragraph{Grobkörniges Multithreading} Führt eine Thread aus, bis eine Verzögerung auftritt, dann
wird nach Round-Robin die Ausführungsreihenfolge geändert. Im Normalfall tritt eine Verzögerung auf,
hier ist es 1 Taktzyklus.\par
\begingroup\setlength\tabcolsep{2pt}
\begin{tabular}{|c|llllllllllllllll|}\hline
  ×  & 1 & 2 & 3 & 4 & 5 & 6 & 7 & 8 & 9 & 10 & 11 & 12 & 13 & 14 & 15 & 16 \\\hline
AE 1 & X1 & X4 & × & Y1 & Y2 & Y4 & × & Z1 & × & X6 & × & Y7 & × & Z5 & × & Z7 \\
AE 2 & X2 & X4 & × &    & Y3 & Y5 & × & Z2 & × & X7 & × & Y8 & × & Z6 & × &    \\
AE 3 & X3 &    & × &    &    & Y6 & × & Z3 & × &    & × &    & × &    & × &    \\
AE 4 &    &    & × &    &    &    & × & Z4 & × &    & × &    & × &    & × &    \\\hline
\end{tabular}
\endgroup

\paragraph{Simultanes Multithreading} Mit Round-Robin Ausführungsreihenfolge werden die Threads
aus alle Ausfüehrungseinheiten aufgeteilt. Dabei gibt es eine Verzögerung falls alle Threads
zu einem Taktzyklus keine Aufgaben haben.\par
\begingroup\setlength\tabcolsep{2pt}
\begin{tabular}{|c|llllllll|}\hline
  ×  & 1 & 2 & 3 & 4 & 5 & 6 & 7 & 8 \\\hline
AE 1 & X1 & Y2 & Z3 & Y4 & Z5 & × & X6 & Z7 \\
AE 2 & X2 & Y3 & Z4 & Y5 & Z6 & × & X7 &    \\
AE 3 & X3 & Z1 & X4 & Y6 &    & × & Y7 &    \\
AE 4 & Y1 & Z2 & X5 &    &    & × & Y8 &    \\\hline
\end{tabular}
\endgroup
