\section{Fehlertoleranz}

\subsection{N-Modulare-Redundanz}
Die Wahrscheinlichkeit eines fehlerfreien Funktionieren eines Systemes $p_{s}$, mit $m$ 
Komponenten, ist abhängig von der Verfügbarkeit der Einzelkomponenten $p_{c}$ und der
Verfügbarkeit des Voters $p_{v}$. Für einen perfekten bzw. nicht vorkommenden Voter
ist $p_{v} = 1$. Die maximale Anzahl der ausfallenden Komponenten $w$, wird falls nicht
gegeben, durch $w = \lceil \cfrac{m}{2} \rceil$ representiert.

\[
    p_{s} = \sum_{i = 0}^{w} \left( \binom{m}{m - 1} \cdot p_{c}^{m - 1} \cdot (1 - p_{c}) \right) \cdot p_{v}
\]

\begin{description}
 \item[2MR]
 Wahrscheinlichkeit des fehlerfreien Funktionierens eines 2MR Systems:\par 
 2 Komponenten mit der Verfügbarkeit $p_{c}$ und einem Voter mit der Verfügbarkeit $p_{v}$:
 \[
    p_{s} = \left(2 p_{c} - p_{c}^{2} \right) \cdot p_{v}
 \]
 \item[3MR]
 Wahrscheinlichkeit des fehlerfreien Funktionierens eines 3MR Systems:\par 
 3 Komponenten mit der Verfügbarkeit $p_{c}$ und einem Voter mit der Verfügbarkeit $p_{v}$:
 \[
    p_{s} = \left(3 p_{c}^{2} - 2 p_{c}^{3} \right) \cdot p_{v}
 \]
 \item[4MR]
 Wahrscheinlichkeit des fehlerfreien Funktionierens eines 4MR Systems:\par 
 4 Komponenten mit der Verfügbarkeit $p_{c}$ und einem Voter mit der Verfügbarkeit $p_{v}$:
 \[
    p_{s} = \left(3 p_{c}^{4} - 8 p_{c}^{3} + 6 p_{c}^{2} \right) \cdot p_{v}
 \]
 \item[5MR]
 Wahrscheinlichkeit des fehlerfreien Funktionierens eines 5MR Systems:\par 
 5 Komponenten mit der Verfügbarkeit $p_{c}$ und einem Voter mit der Verfügbarkeit $p_{v}$:
 \[
    p_{s} = \left(6 p_{c}^{5} - 15 p_{c}^{4} + 10 p_{c}^{3} \right) \cdot p_{v}
 \]
\end{description}
