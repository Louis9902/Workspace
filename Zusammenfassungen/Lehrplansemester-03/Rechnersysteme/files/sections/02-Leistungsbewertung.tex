\section{Leistungsbewertung}

\paragraph{Kenngrößen}
\begin{itemize}
 \item \textbf{MIPS} millions of instructions per second
 \item \textbf{FLOPS} floating point operations per second
\end{itemize}

\important
MIPS-Vergleiche ergeben lediglich Sinn inerhalb der gleichen ISA (Instruction Set Architecture).
\[
    \text{MIPS} = \cfrac{10^3}{\text{Instruktionszeit in ns}}
\]

\subsection{Amdahlsches Gesetz}
Die Beschleunigung eines Systems wird berechnet durch ein vergleich der Zeit:
\[
    \text{Beschleunigung} = \cfrac{\text{Zeit vor der Beschleunigung}}{\text{Zeit nach der Beschleunigung}}
\]
Damit kann man folgende Gesetzmäßigkeit aufstellen: 
\[
    S = \cfrac{1}{(1-p)+\cfrac{p}{s}}
\]
Aus diesem Zusammenhang kann man nun auch folgendes herleiten:
\[
    \cfrac{T_{0}}{T_{S}} = \cfrac{1}{(1-p)+\cfrac{p}{s}}
\]
\begin{definitions}
    $$S$$ & (Gesamt-)Beschleunigung des Programmes \\
    $$s$$ & Beschleunigung des Pro­gramm­teils welches von der Verbesserung profitiert \\
    $$p$$ & Anteil der Zeit, in dem die Verbesserung benutzt wird \\
    $$T_{0}$$ & Zeit vor der Beschleunigung \\
    $$T_{S}$$ & Zeit nach der Beschleunigung
\end{definitions}


