\section{Paritätsprüfung}

\subsection{Paritätsbit}
Die Wahrscheinlichkeit das ein Wort der Länge $n$ wirklich fehlerfrei ist, wird bereits
durch die Fehlererkennung mittels eines Paritätsbits erhöht. Dabei ist die 
\emph{Wahrscheinlichkeit für ein} korrektes Bit $p$, wobei gilt das $0<p<1$ ist. 
Die Bitfolge mit der Länge $n+1$ wird dann als korrekt angesehen wenn alle $n+1$ oder
$n$ Bits korrekt sind. Die Wahrscheinlichkeit für das Wort mit der Länge $n$ und einem
Paritätsbit wird wiefolt berechnet:
\begin{align*}
    P_{\text{parity}}   &= p^{n+1}+\binom{n+1}{n}\cdot p^{n}\cdot\left(1-p\right)    \\
                        &= p^{n+1}+\left(n+1\right)\cdot\left(p^{n}-p^{n+1}\right)    \\
                        &= \left(n+1\right)\cdot p^{n}-n\cdot p^{n + 1}
\end{align*}

Für ein Wort der Länge $n$ ist die Wahrscheinlichkeit das es korrekt angesehen wird:
\[
P_{\text{no\_parity}} = p^{n}
\]
