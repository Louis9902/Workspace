\documentclass
[
	8pt,		% font size
	ngerman,	% hyphenation and more
	a4paper,	% paper size
	landscape,	% orientation
	final		% document status (final/draft)
]{extarticle}

% adjust language %
\usepackage[ngerman]{babel}
% integration of speacial characters %
\usepackage[utf8]{inputenc}
\usepackage[T1]{fontenc}
\usepackage{textcomp}
% adjust page layout %
\usepackage{titlesec}
\usepackage{fancyhdr}
% --------------------------- %
\usepackage{multicol}
\usepackage{multirow}
% --------------------------- %
\usepackage{setspace}
\usepackage{geometry}
\usepackage{adjustbox}
% adjust colors %
\usepackage{color}
% integration of mathematical symbols %
\usepackage{amsmath}
\usepackage{amssymb}
\usepackage{amsthm}
\usepackage{mathtools}
% integration of source code %
\usepackage{listings}
% adjust enumerations %
\usepackage{enumitem}
% adjust tables %
\usepackage{tabularx}
% integration of graphics %
\usepackage{graphicx}
% create graphics %
\usepackage{tikz}
\usetikzlibrary{automata, positioning, arrows}
% integration of seperate files %
\usepackage{standalone}

% colors %
\usepackage{color}
\definecolor{cmark}{HTML}{008000}
\definecolor{xmark}{HTML}{E32636}

% tick and cross symbols %
\usepackage{pifont}
\newcommand{\cmark}{\textcolor{cmark}{\ding{51}}}
\newcommand{\xmark}{\textcolor{xmark}{\ding{55}}}

\newcommand*\circlenummber[1]{\tikz[baseline=(char.base)]{\node[shape=circle,draw,inner sep=0.1ex] (char) {#1};}}

\DeclareMathOperator{\avg}{avg}

% ========== INFORMATION ========== %
\def\name{Louis Seubert}
\def\prefix{Zusammenfassung}
\def\lecture{Rechnersysteme}
\def\maxpages{6}

% ============== ADJUSTMENTS ============== %
% adjust graphics path %
\graphicspath
{
	{figures/}
}
% adjust page layout %
\geometry
{
	left=0.55cm,
	right=0.55cm,
	top=1.10cm,
	bottom=0.55cm,
	headsep=2mm
}
% adjust source code view %
\lstset
{
	basicstyle=\ttfamily\footnotesize,
	columns=fullflexible,
	numbers=left,						% where to put the line-numbers
	numberstyle=\tiny,  				% the style that is used for the line-numbers
	stepnumber=1,
	numbersep=5pt,						% how far the line-numbers are from the code
	showspaces=false,					% show spaces adding particular underscores
	showstringspaces=false,				% underline spaces within strings
	showtabs=false,						% show tabs within strings adding particular underscores
	frame=none,							% adds a frame around the code
	tabsize=2,							% sets default tabsize to 2 spaces
	captionpos=b,						% sets the caption-position to bottom
	breaklines=true,					% sets automatic line breaking
	breakatwhitespace=false,			% sets if automatic breaks should only happen at whitespace
	xleftmargin=10pt					% left margin to prevent number clipping
}

% make header and footer %
\pagestyle{fancy}
\fancyhead{} % clear header
\fancyhead[L]{\prefix\;\lecture}
\fancyhead[R]{\thepage\;--\;\maxpages}
\fancyhead[C]{\name}
\fancyfoot{} % clear footer

% configure document %
\setitemize{leftmargin=15pt}
\setenumerate{leftmargin=15pt}
\setlist{itemsep=1pt,parsep=1pt,noitemsep}

\setlength{\parindent}{0pt}
\setlength{\parskip}{0pt}
\setlength{\topskip}{10pt}
% set column seperator %
\setlength{\columnseprule}{0.5pt}

% change style %
\titleformat*{\section}{\large\bfseries}
\titlespacing*{\section}{0pt}{4pt}{0pt}

\titleformat*{\subsection}{\normalsize\bfseries}
\titlespacing*{\subsection}{0pt}{4pt}{0pt}

\titleformat*{\subsubsection}{\normalsize\bfseries}
\titlespacing*{\subsubsection}{0pt}{4pt}{0pt}

\titleformat*{\paragraph}{\normalsize\bfseries}
\titlespacing{\paragraph}{0pt}{.5em}{.5em}

\titleformat*{\subparagraph}{\small\bfseries}
\titlespacing*{\subparagraph}{0pt}{.5em}{.5em}

\newcommand*\important{\par\vspace{\abovedisplayskip}\textbf{Wichtig:}\par}
\newcommand*\example{\par\vspace{\abovedisplayskip}\textbf{Beispiel:}\par}
\newcommand{\includefigure}[1]{\begin{center}\input{#1}\end{center}}

\newenvironment{definitions}{
    \par\vspace{\abovedisplayshortskip}\noindent
    \tabularx{\columnwidth}{>{$}l<{$} @{${}={}$} >{\raggedright\arraybackslash}X}
}{\endtabularx\par\vspace{\belowdisplayshortskip}}

\begin{document}
\begin{multicols*}{4}
% ======================================================================== %
\section{Einheiten, Größen}
% ------------------------------------------------------------------------ %
\subsection{Datenmengen}
\begin{center}
	\documentclass[12pt]{standalone}

\usepackage{amssymb}
\usepackage{multirow}
\usepackage{multicol}
\usepackage{adjustbox}

\begin{document}
\begin{tabular}{|c|c|c|c|c|c|c|}
	\hline\rule{0pt}{3ex}
	\space                                                                       & \multicolumn{6}{c|}{Dezimalpräfix $\longrightarrow$}                                                                      \\
	\hline\rule{0pt}{3ex}
	\multirow{6}{*}{\rotatebox[origin=center]{90}{$\longleftarrow$ Binärpräfix}} & \space                                               & \textbf{B} & \textbf{kB} & \textbf{MB} & \textbf{GB} & \textbf{TB} \\
	\cline{2-7}\rule{0pt}{3ex}                                                   & \textbf{B}                                           & 1          & $10^{3}$    & $10^{6}$    & $10^{9}$    & $10^{12}$   \\
	\cline{2-7}\rule{0pt}{3ex}                                                   & \textbf{KiB}                                         & $2^{10}$   & 1           & $10^{3}$    & $10^{6}$    & $10^{9}$    \\
	\cline{2-7}\rule{0pt}{3ex}                                                   & \textbf{MiB}                                         & $2^{20}$   & $2^{10}$    & 1           & $10^{3}$    & $10^{6}$    \\
	\cline{2-7}\rule{0pt}{3ex}                                                   & \textbf{GiB}                                         & $2^{30}$   & $2^{20}$    & $2^{10}$    & 1           & $10^{3}$    \\
	\cline{2-7}\rule{0pt}{3ex}                                                   & \textbf{TiB}                                         & $2^{40}$   & $2^{30}$    & $2^{20}$    & $2^{10}$    & 1           \\
	\hline
\end{tabular}
\end{document}
\end{center}
% ------------------------------------------------------------------------ %
\subsection{Frquenz \(f\)}
\[f = \cfrac{1}{T}\]

% ======================================================================== %
\section{Leistungsbewertung}
% ------------------------------------------------------------------------ %
\subsection{Kenngrößen}
\begin{itemize}
	\item \textbf{MIPS} millions of instructions per second
	\item \textbf{FLOPS} floating point operations per second
\end{itemize}
\important
MIPS-Vergleiche ergeben lediglich Sinn inerhalb der gleichen ISA
(Instruction Set Architecture).
\[\text{MIPS} = \cfrac{10^3}{\text{Instruktionszeit in ns}}\]
% ------------------------------------------------------------------------ %
\subsection{Amdahlsches Gesetz}
Die Beschleunigung eines Systems wird berechnet durch ein vergleich der Zeit:
\[\text{Beschleunigung} = \cfrac{\text{Zeit vor der Beschleunigung}}{\text{Zeit nach der Beschleunigung}}\]
Damit kann man folgende Gesetzmäßigkeit aufstellen:
\[S = \cfrac{1}{(1-p)+\cfrac{p}{s}}\]
Aus diesem Zusammenhang kann man nun auch folgendes herleiten:
\[\cfrac{T_{0}}{T_{S}} = \cfrac{1}{(1-p)+\cfrac{p}{s}}\]
\begin{definitions}
	$$S$$ & (Gesamt-)Beschleunigung des Programmes \\
	$$s$$ & Beschleunigung des Pro­gramm­teils welches von der Verbesserung profitiert \\
	$$p$$ & Anteil der Zeit, in dem die Verbesserung benutzt wird \\
	$$T_{0}$$ & Zeit vor der Beschleunigung \\
	$$T_{S}$$ & Zeit nach der Beschleunigung
\end{definitions}
Wenn sich z.B. die Anzahl der Prozessor Kerne ver-\(n\)-facht dann ist die
Beschleunigung des Pro­gramm­teils \(\frac{1}{n}\) wenn es vorher auf einem
Kern lief.

% ======================================================================== %
\section{Fehlertoleranz}
% ------------------------------------------------------------------------ %
\subsection{N-Modulare-Redundanz}
Die Wahrscheinlichkeit eines fehlerfreien Funktionieren eines Systemes $p_{s}$, mit $m$
Komponenten, ist abhängig von der Verfügbarkeit der Einzelkomponenten $p_{c}$ und der
Verfügbarkeit des Voters $p_{v}$. Für einen perfekten bzw. nicht vorkommenden Voter
ist $p_{v} = 1$. Die maximale Anzahl der ausfallenden Komponenten $w$, wird falls nicht
gegeben, durch $w = \lceil \frac{m}{2} \rceil$ representiert.
\[p_{s} = \sum_{i = 0}^{w} \left( \binom{m}{m - i} \cdot p_{c}^{m - i} \cdot (1 - p_{c})^{i} \right) \cdot p_{v}\]

\begin{description}
	\item[2MR]
	      Wahrscheinlichkeit des fehlerfreien Funktionierens eines 2MR Systems:\par
	      2 Komponenten mit der Verfügbarkeit $p_{c}$ und einem Voter mit der Verfügbarkeit $p_{v}$:
	      \[p_{s} = \left(2 p_{c} - p_{c}^{2} \right) \cdot p_{v}\]
	\item[3MR]
	      Wahrscheinlichkeit des fehlerfreien Funktionierens eines 3MR Systems:\par
	      3 Komponenten mit der Verfügbarkeit $p_{c}$ und einem Voter mit der Verfügbarkeit $p_{v}$:
	      \[p_{s} = \left(3 p_{c}^{2} - 2 p_{c}^{3} \right) \cdot p_{v}\]
	\item[4MR]
	      Wahrscheinlichkeit des fehlerfreien Funktionierens eines 4MR Systems:\par
	      4 Komponenten mit der Verfügbarkeit $p_{c}$ und einem Voter mit der Verfügbarkeit $p_{v}$:
	      \[p_{s} = \left(3 p_{c}^{4} - 8 p_{c}^{3} + 6 p_{c}^{2} \right) \cdot p_{v}\]
	\item[5MR]
	      Wahrscheinlichkeit des fehlerfreien Funktionierens eines 5MR Systems:\par
	      5 Komponenten mit der Verfügbarkeit $p_{c}$ und einem Voter mit der Verfügbarkeit $p_{v}$:
	      \[p_{s} = \left(6 p_{c}^{5} - 15 p_{c}^{4} + 10 p_{c}^{3} \right) \cdot p_{v}\]
\end{description}

% ======================================================================== %
\section{Paritätsprüfung}
% ------------------------------------------------------------------------ %
\subsection{Paritätsbit}
Die Wahrscheinlichkeit das ein Wort der Länge $n$ wirklich fehlerfrei ist, wird bereits
durch die Fehlererkennung mittels eines Paritätsbits erhöht. Dabei ist die
\emph{Wahrscheinlichkeit für ein} korrektes Bit $p$, wobei gilt das $0<p<1$ ist.
Die Bitfolge mit der Länge $n+1$ wird dann als korrekt angesehen wenn alle $n+1$ oder
$n$ Bits korrekt sind. Die Wahrscheinlichkeit für das Wort mit der Länge $n$ und einem
Paritätsbit wird wiefolt berechnet:
\begin{align*}
	P_{\text{parity}} & = p^{n+1}+\binom{n+1}{n}\cdot p^{n}\cdot\left(1-p\right)  \\
	                  & = p^{n+1}+\left(n+1\right)\cdot\left(p^{n}-p^{n+1}\right) \\
	                  & = \left(n+1\right)\cdot p^{n}-n\cdot p^{n + 1}
\end{align*}

Für ein Wort der Länge $n$ ist die Wahrscheinlichkeit das es korrekt angesehen wird:
\[P_{\text{no\_parity}} = p^{n}\]

% ======================================================================== %
\section{Hamming-Code}
% ------------------------------------------------------------------------ %
\subsection{Anzahl der Prüf- und Datenbits}
Für ein Wort der länge $n$ mit $m$ Datenbits benötigt man $r$ Prüfbits um einen 1-Bit-Fehler
pro Wort zu korrigiren.
Daraus ergibt sich die allgemeine Ungleichung: \[2^{r} \geq m + r + 1\]
Und für einen \emph{perfekten} Hamming-Code ergibt sich: \[2^{r} = m + r + 1\]
Der \emph{perfekte} Hamming-Code besitzt die Wortlänge $2^{r} - 1$.
% ------------------------------------------------------------------------ %
\subsection{Hamming-Abstand}
Der Hamming-Abstand gibt die Anzahl der Bits zwischen zwei beliebige Wörtern aus dem Code.
\example
Die zwei Wörter $\texttt{1100100100}_2$ und $\texttt{1001100011}_2$ haben den
Hamming-Abstand $h = 5$.

\subsubsection*{Hamming-Abstand eines Codes}
Der Hamming-Abstand eines Codes ist die kleinstmögliche Anzahl an Bits, die man verändern
muss um ein neues gültiges Wort aus dem vorliegenden Code zu bekommen.
\example
\begin{center}
	\begin{minipage}{.35\linewidth}
		\begin{align*}
			x = \texttt{00110} \\
			y = \texttt{00101} \\
			z = \texttt{01110}
		\end{align*}
	\end{minipage}
	\begin{minipage}{.35\linewidth}
		\begin{eqnarray*}
			h_{xy} &=& 2    \\
			h_{xz} &=& 1    \\
			h_{yz} &=& 3
		\end{eqnarray*}
	\end{minipage}
\end{center} \par
Da der kleinste Abstand 1 ist, ist der Hamming-Abstand des Codes ebenfalls 1.
% ------------------------------------------------------------------------ %
\subsection{Fehlererkennung}
Um $d$ Bitfehler zu erkennen, braucht man den Hamming-Abstand $h$:
\begin{eqnarray*}
	h &=& d + 1 \\
	d &=& h - 1
\end{eqnarray*}
% ------------------------------------------------------------------------ %
\subsection{Fehlerkorrektur}
Um $d$ Bitfehler zu korrigiren, braucht man den Hamming-Abstand $h$:
\begin{eqnarray*}
	h &=& 2d + 1 \\
	d &=& \left\lfloor\cfrac{h - 1}{2}\right\rfloor
\end{eqnarray*}
% ------------------------------------------------------------------------ %
\subsection{Bündelfehlerkorrektur}
Um mehrere aufeinanderfolgende Bitfehler zu korrigiren ordnet man die $k$ mit Hamming
codierten Wörter der Länge $n$ in einer $k \cdot n$ Matrix an. \emph{Dabei ist zu
	beachten das nun zuerst die erste Spalte der Matrix übertragen bzw. gespeichert wird.}\par
Somit kann man einen \textbf{maximal} einen Bündelfehler der Länge $k$ Bits korrigiren,
dabei ist bei $k$ aufeinanderfolgenden fehlerhaften Bits maximal 1 Bit pro Wort fehlerhaft.

\example
\includefigure{./Documents/Graphics/Hamming-Abbildung-1.tex}
Die übertragenen Bits lauten:\par
\texttt{0111000 0011111 1111111 1100001 0111111 0000000 1011110 0100101 0011011 0000011 0100101}
% ------------------------------------------------------------------------ %
\subsection{Hamming-Algorithmus}
\subsubsection*{Paritäts-Kontrollbits}
Paritäts-Kontrollbits $p$ an Zweierpotenz Positionen
\begin{itemize}
	\item Paritätsbit $p_{1}$ prüft 1, 3, 5, 7, 9, 11, 13, 15, 17, 19, 21
	\item Paritätsbit $p_{2}$ prüft 2, 3, 6, 7, 10, 11, 14, 15, 18, 19
	\item Paritätsbit $p_{4}$ prüft 4, 5, 6, 7, 12, 13, 14, 15, 20, 21
	\item Paritätsbit $p_{8}$ prüft 8, 9, 10, 11, 12, 13, 14, 15
	\item Paritätsbit $p_{16}$ prüft 16, 17, 18, 19, 20, 21
\end{itemize}

\example
\includefigure{./Documents/Graphics/Hamming-Abbildung-2.tex}
% ------------------------------------------------------------------------ %
\subsection{Erweiterungen}
\subsubsection*{SECDED}
\textbf{S}ingle \textbf{E}rror \textbf{C}orrection \textbf{D}ouble \textbf{E}rror
\textbf{D}etection füget ein weiteres Paritätsbit an der Position \texttt{0} welches
die gesamten anderen Bits auf Parität überprüft. \par
\begin{itemize}
	\item Hamming-Prüfbits \cmark \quad Extra-Prüfbit \cmark\par
	      $\longrightarrow$ fehlerfrei oder mehr als 2 Fehler
	\item Hamming-Prüfbits \xmark \quad Extra-Prüfbit \xmark\par
	      $\longrightarrow$ 1-Bitfehler, korrigirbar mit Hamming
	\item Hamming-Prüfbits \xmark \quad Extra-Prüfbit \cmark\par
	      $\longrightarrow$ 2-Bitfehler, nicht lokalisier- bzw. korrigirbar
\end{itemize}


\newcommand{\microinst}{\ensuremath{\mu\text{-Instruktionen}}}
% ======================================================================== %
\section{IJVM}
% ------------------------------------------------------------------------ %
\subsection{Ausführungsdauer}
\[
	\text{Ausführungszeit} = \cfrac{\microinst}{\text{Takt}}
\]
\important Bei der Auswärtung der \microinst ist darauf zu achten dass der
Programm fluss beachtet wird, denn je nach Aufbau des Programmes kann es bei
bedingten Sprüngen in der Architektur sein das Befehle eine unterscheidliche
Anzahl an \microinst besitzt. Ebenfalls muss beachtet werden ob inerhalb der
\microinst auch sprünge stattfinden, welche zusätzliche Operation zum selben
Takt haben.
% ------------------------------------------------------------------------ %
\subsection{Anweisungsabhängigkeit}
\begin{description}
	\item[Read after Write (RAW)]
	      Instruktion \(2\) versucht aus dem Register einen Wert zu lesen
	      bevor dieser durch Instruktion \(1\) geschrieben wurde. \par
	      Diese Art der Abhängigkeit ist entscheidend für Aussagen über die
	      Mindestanzahl der benötigten Taktzyklen. Für die Aufteilung der
	      einzelnen Instruktionen auf die verscheiedenen
	      Ausführungseinheiten eines superskalaren Prozessors werden
	      Abhängigkeitskette gebildet, wobei die Anzahl der Glieder in der
	      längsten Kette der Mindestanzahl der benötigten Taktzyklen
	      entspricht.
	      \begin{align}
		      \textbf{R1} & = \text{R2} + \text{R3}   \\
		      \text{R4}   & = \textbf{R1} + \text{R5}
	      \end{align}
	      \setcounter{equation}{0}

	\item[Write after Read (WAR)]
	      Instruktion \(2\) versucht in ein Register einen Wert zu schreiben
	      bevor es von Instruktion \(1\) gelesen wurde.
	      \begin{align}
		      \text{R1}   & = \textbf{R2} + \text{R3} \\
		      \textbf{R2} & = \text{R5} + \text{R6}
	      \end{align}
	      \setcounter{equation}{0}

	\item[Write after Write (WAW)]
	      Instruktion \(2\) versucht in ein Register einen Wert zu schreiben
	      bevor in dieses ein Wert von Instruktion \(1\) geschrieben wurde.
	      \begin{align}
		      \textbf{R1} & = \text{R2} + \text{R3} \\
		      \textbf{R1} & = \text{R5} + \text{R6}
	      \end{align}
	      \setcounter{equation}{0}
\end{description}
% ------------------------------------------------------------------------ %
\subsection{Adressierungsarten}
\begin{description}
	\item[Unmittelbare Adressierung]
	      Bei der unmittelbaren Adressierung folgt ein Operand direkt dem
	      Opcode. Er wird als Konstante betrachtet. \\
	      $\texttt{LOAD 20}\rightarrow\texttt{20}$

	\item[Direkte Adressierung]
	      Bei der direkten Adressierung wird der Operand der dem Befehl
	      folgt oder das angegebene Register als Adresse angesehen. \\
	      $\texttt{LOAD 20}\rightarrow\texttt{40}$

	\item[Indirekte Adressierung]
	      Bei der indirekten Adressierung enthält die Speicherzelle dagegen
	      erneut eine Adresse und erst die Speicherzelle die durch diese
	      Adresse angesprochen wird den Wert selbst. \\
	      $\texttt{LOAD 20}\rightarrow\texttt{60}$ \\
	      $\texttt{LOAD R2}\rightarrow\texttt{40}$

	\item[Registeradressierung]
	      Bei der Registeradressierung ist der Operand ein Register, welches
	      den Wert selbst beinhaltet. \\
	      $\texttt{LOAD R2}\rightarrow\texttt{20}$

	\item[Indizierte Adressierung]
	      Dies entspricht der direkten Adressierung, nur wird vor dem
	      Zugriff zu der Adresse noch ein fester Wert addiert und das
	      Resultat als Adresse interpretiert. \\
	      $\texttt{LOAD (R1 + 40)}\rightarrow\texttt{70}$ \\
	      Anstelle des festen Wert kann auch ein Indexregister zu einem
	      Basisregister addiert werden, dies liefert dann die Adresse für
	      den direkten oder indirekten Zugriff. \\
	      $\texttt{LOAD (R1 + R2)}\rightarrow\texttt{50}$
\end{description}
\emph{Beispielwerte:}
\begin{center}
	\begin{tabular}{|r|l|c|r|l|}\cline{1-2}\cline{4-5}
		Register    & Wert        &  & Adresse     & Wert        \\\cline{1-2}\cline{4-5}
		\texttt{R1} & \texttt{10} &  & \texttt{20} & \texttt{40} \\\cline{1-2}\cline{4-5}
		\texttt{R2} & \texttt{20} &  & \texttt{30} & \texttt{50} \\\cline{1-2}\cline{4-5}
		            &             &  & \texttt{40} & \texttt{60} \\\cline{1-2}\cline{4-5}
		            &             &  & \texttt{50} & \texttt{70} \\\cline{1-2}\cline{4-5}
	\end{tabular}
\end{center}


% ======================================================================== %
\section{Hard Disk Drive}
% ------------------------------------------------------------------------ %
\subsection{Begriffe}
\begin{description}
	\item[Spur]
	      Ein Kreis auf der Festplatte auf dem die Sektoren nacheinander
	      liegen. Die Spuren sind konzentrisch auf der Festplatte gelegt.

	\item[Sektor]
	      Ein Stück aus einer Spur mit fester Breite. Die Sektorlänge ist
	      z.B. in Bytes ($512$ oder $4096$ Bytes) oder Mikrometer angegeben.
	      Jeder Sektor enthält zudem noch $40$ bis $100$ Bytes für
	      Verwaltungsinformationen (\emph{Präambel}) und
	      Fehlererkennung/-korrektur.

	\item[Zylinder]
	      Wenn die Festplatte aus mehreren übereinander liegenden Scheiben
	      besteht, besteht ein Zylinder allen Spuren die genau übereinander
	      auf den Scheiben liegen. Wenn es z.B. 2 Scheiben gibt, dann sind
	      alle $2$ Spuren die ganz außen auf jeder Scheibe liegen ein
	      Zylinder; genauso machen alle Spuren Nr. $16$ auf jeder Scheibe
	      zusammen ein Zylinder aus: Zylinder $16$. Ein Zylinder ist also
	      eine Erweiterung der Spur auf vertikalen Ebene wenn es mehrere
	      Scheiben gibt. Es folgt, auf einem Zylinder der aus $n$
	      übereinander liegenden Spuren besteht und wenn jede dieser Spuren
	      $k$ Bytes hat, dann hat ein Zylinder $n \cdot k$ Bytes. Folgt
	      auch, dass eine Angabe wie $32$ Zylinder bedeutet dass jede
	      Scheibe $32$ Spuren hat.
\end{description}
% ------------------------------------------------------------------------ %
\subsection{Datenübertragungszeit}
Die mittlere Datenübertragungszeit $T_{\text{average}}$ einer Festplatte
hinsichtlich der auf einer Spur befindlichen Daten kann vereinfacht mittels
der folgenden Funktion dargestellt werden:
\[T_{\text{average}} := T_{s} + T_{r} + T_{t}\]
Die mittlere Positionierungszeit $T_{s}$ setzt sich dabei zusammen aus der
Positionierungszeit $MT_{\text{sr}}$ des Schreib/Lese-Kopfs in die richtige
radiale Position zum Lesen/Schreiben und der Zeit $MT_{\text{sc}}$ zum
wechseln auf benachbarte Spuren $S_{c}$, daraus leiten sich folgenden
Funktionen ab:
\begin{align*}
	T_{s}         & := T_{\text{sr}} + T_{\text{sc}} \\
	T_{\text{sr}} & := MT_{\text{sr}}\cdot (S-S_{c}) \\
	T_{\text{sc}} & := MT_{\text{sc}}\cdot S_{c}
\end{align*}
Sollten jedoch alle Spuren $S$ bis auf die erste benachbart sein, dann gilt
für die Positionierungszeit folgenden Funktion:
\[T_{s} := MT_{\text{sr}} + MT_{\text{sc}}\cdot (S-1)\]
Der Drehverzug $T_{r}$ fällt bei jedem Spurwechsel an und ist die
Verzögerung, welche anfällt bis sich der gesuchte Sektor auf der rotierenden
Festplatte unter dem Schreib/Lese-Kopf befindet. Die Drehverzugszeit ist
dabei abhängig von der Rotationsgeschwindigkeit $v$ welche in \emph{RPM}
angegeben wird. \[T_{r} := \left(\cfrac{60\cdot 1000}{2v}\right)\cdot S\]
Die Übertragungszeit $T_{t}$ ist die eigentliche Zeit welche benötigt wird
und die Daten zu übertragen; diese ist abhängig von der
Rotationsgeschwindigkeit $v$ sowie der Anzahl der genutzten Sektoren
$N_{u}$ pro Spur und der Gesamtanzahl der Sektoren $N_{t}$ pro Spur.
\[T_{t} := \left(\cfrac{60\cdot 1000}{v}\right)\cdot \cfrac{N_{u}}{N_{t}}\cdot S\]
Die Anzahl der benötigten Spuren lässt wie folgt berechnen:
\[\text{Spuren} = \cfrac{\text{Datei}_\text{Gr}}{\text{Oberflächen}_\text{Anz}\cdot \text{Sektoren}_\text{Anz}\cdot \text{Sektor}_\text{Gr}}\]
% ------------------------------------------------------------------------ %
\subsection{Speicherdirektzugriff}
Der Speicherdirektzugriff ermöglicht es angeschlossenen Peripheriegeräten
ohne Umweg über die CPU mit dem Arbeitsspeicher zu kommuniziren. Ein Vorteil
des Speicherdirektzugriffs ist die schneller Datenübertargung bei einer
gleichzeitigen Entlastung des Prozessors. Der DMA-Controller muss jedoch die
Daten zwangsläufig über das gleiche Bussystem übertragen wie die CPU. \par
Die Buszyklen $C$ welche für einen $n$-Bit-DMA-Transfer benötigten werden
kann mittels folgender Funktion, in Abhängigkeit von der Umdrehungsdauer
$T_{r}$ (in ms), der Sektorgröße $S$ und der Sektorenanzahl $A$,ermittelt
werden: \[C = \left\lceil\cfrac{A\cdot S\cdot 8}{n\cdot T_{r}}\right\rceil\]

% ======================================================================== %
\section{Solid State Drive}
\begin{definitions}
	$$C$$ & Lösch/Schreib-Zyklen pro Speicherzelle                                  \\
	$$T_{L}$$ & Lebensdauer                                                         \\
	$$B_{G}$$ & Gesamtanzahl der Blöcke                                             \\
	$$B_{T}$$ & Anzahl der Blöcke welche pro Zeiteinheit geschrieben werden         \\
	$$P_{D}$$ & Prozentanteil der dynamischen Daten                                 \\
	$$P_{S}$$ & Prozentanteil der reservierten Blöcke für das Bad Block Management
\end{definitions}
\[B_{G} = \cfrac{\text{Kapazität}}{\text{Blockgröße}} \cdot ( 1 - P_{S} )\]
\important
Wenn es heißt: \emph{Eine typische Datei, die zu den Dynamischen Daten gerechnet
	wird, belegt im Mittel $x$ Blöcke ...}, dann muss wie flogt vorgegangen
werden:\par
$A_{\text{DBF}} = \text{Anzahl Dynamischener Blöcke pro Datei}$
\[\text{Anzahl}_\text{Datein} = \cfrac{\text{Anzahl}_\text{Dynamischener Blöcke}}{A_{\text{DBF}}}\]
\[B_{T} = A_{\text{DBF}} \cdot \text{Anz}_\text{Datein} \cdot \text{Zeiteinheit}\]
% ------------------------------------------------------------------------ %
\subsection{Static Wear Levelling}
\[T_{L} = \cfrac{C \cdot B_{G}}{B_{T}}\]
% ------------------------------------------------------------------------ %
\subsection{Dynamic Wear Levelling}
\[T_{L} = \cfrac{C \cdot ( B_{G} \cdot P_{D} )}{B_{T}}\]
% ======================================================================== %
\section{Caches}
% ------------------------------------------------------------------------ %
\subsection{Zugriffszeit}
Die Trefferwahrscheinlichkeit für einen Wert in eimem Level-$i$-Cache
berechnet siech wie folgt:
\begin{align*}
	P(\text{L1})  & = p_{\text{L1}}                                                                                \\
	P(\text{L2})  & = p_{\text{L2}} \cdot (\underbracket{1 - P(\text{L1})}_{\text{L1 Miss}})                       \\
	P(\text{L3})  & = p_{\text{L3}} \cdot (\underbracket{1 - P(\text{L1}) - P(\text{L2})}_{\text{L1 und L2 Miss}}) \\
	P(\text{MEM}) & = 1 - (\underbracket{P(\text{L1}) - P(\text{L2}) - P(\text{L3})}_{\text{nicht im Cache}})
\end{align*}
Dabei bezeichnet $p_{\text{L}i}$ den Prozentsatz der Speicher-Lesezugriff,
die Cache-Treffer des Level-$i$-Caches sind. Die mittlere Zugriffszeit auf
Werte aus dem Hauptspeicher bzw. dem Cache lässt sich dann wie folgt
berechen:
\[T_{\avg} = P(\text{MEM}) \cdot \left( T_{i - 1} + T_{\text{MEM}} \right) + \sum\left( T_{i} \cdot P(\text{L}_{i}) \right)\]
Wobei $T_{i}$ die Zugriffszeit für den Level-$i$-Cache in Nanosekunden ist,
$T_{\text{MEM}}$ ist die Zugriffszeit auf den Hauptspeicher und $T_{i - 1}$
ist der letzte Cache vor dem Hauptspeicher.
\important
$T_{i - 1} \neq 0$, wenn erst \emph{nach} dem letzten Cache auf den
Hauptspeicher zugegriffen wird. Bei einem parallelen Zugriff auf die Caches
und den Hauptspeicher gilt: $T_{i - 1} = 0$.
% ------------------------------------------------------------------------ %
\subsection{Virtuelle Speicheradresse}
\begin{center}
	\begin{tikzpicture}[scale=0.4]
		\draw (00,00) rectangle (05,01) node[pos=0.5] {\small\textbf{TAG}};
		\draw (05,00) rectangle (09,01) node[pos=0.5] {\small\textbf{LINE}};
		\draw (09,00) rectangle (12,01) node[pos=0.5] {\small\textbf{WORD}};
		\draw (12,00) rectangle (15,01) node[pos=0.5] {\small\textbf{BYTE}};
	\end{tikzpicture}
\end{center}
\begin{description}
	\item[TAG] Das Feld Tag beshetht aus einem eindeutigen n-Bit-Wert. Er
	      kennzeichnet die entsprechenden Zeile im Hauptspeicher, woher die
	      Daten stammen. Die Anzahl der Bist wird meist so gewählt das eine
	      volle Speicheradresse (32/64 Bits) entsteht.
	      \[\text{Bits}_\texttt{TAG} = \text{Bits}_\texttt{ADDRESS} - \text{Bits}_\texttt{LINE} - \text{Bits}_\texttt{WORD}\]
	\item[LINE]
	      Das Line Feld gibt an, in welcher Zeile sich die entsprechenden
	      Daten befinden, falls diese im Cache vorhanden sein sollten.
	      \[\text{Bits}_\texttt{LINE} = \left\lceil \log_{2}\left(\cfrac{\text{Cachegröße}}{\text{Cachezeilengröße}}\right) \right\rceil\]
	\item[WORD] Mit dem Word Feld wird angegeben welches Wort aus der Cache
	      Zeile angefordert wird. Dies fällt oftmals zusammen mit dem
	      \emph{BYTE} Feld, da der Cache lediglich eine Byte-Adressierung
	      zulässt.
	\item[BYTE] Das Byte Feld gibt an welches Byte aus einem Speicherwort
	      geldaen werden soll. Dies kann bei einem Cache der lediglich eine
	      Byte-Adressierung zulässt auch mit dem \emph{WORD} Feld zusammengefast
	      werden.
	      \[\text{Bits}_\texttt{WORD} = \left\lceil \log_{2}\left(\text{Cachezeilengröße}\right) \right\rceil\]
\end{description}
% ~~~~~~~~~~~~~~~~~~~~~~~~~~~~~~~~~~~~~~~~~~~~~~~~~~~~~~~~~~~~~~~~~~~~~~~~ %
\subsubsection*{N-Weg-Cache}
Ermöglicht es für eine Speicheradresse mehrere Einträge abzulegen. Dies wird
benötig wenn ein Programm häufig Wörter an zwei weit auseinander liegenden
Adressen holt und dadurch ständig Konflikte auftreten welche möglicherweße
die vorherige Referenz aus dem Cache verdrängt. Ein Zusammfassung $n$
solcher Cache-Einträge nennt man \emph{Menge}.
\[ \text{Anzahl}_\text{Mengen} = \cfrac{\text{Cachegröße}}{\text{Cachezeilengröße} \cdot \text{Assoziativität}}\]
Da die Kapazität des Caches zunimmt jedoch nicht die Anzahl der
adressierbaren Cachezeilen, dadurch muss die Anzahl der benötigten Bist für
die Cachezeilen wiefolgt lauten:
\[\text{Bits}_\texttt{LINE} = \left\lceil log_{2}\left( \text{Anzahl}_\text{Mengen} \right) \right\rceil\]
% ------------------------------------------------------------------------ %
\subsection{Schreib-Operationen}
Schreibt der Prozess ein Wort und das Wort befindet sich im Cache, muss der
Cache-Eintrag aktualisirt werden, heizu gibt es folgende möglichkeiten:
\begin{enumerate}
	\item Cache-Treffer
	      \begin{enumerate}
		      \item\textbf{Write Through} Die modifizierten Daten werden
		            sofort in den Hauptspeicher zurück geschrieben.
		      \item\textbf{Write Back} Die modifizierten Daten werden erst
		            in den Hauptspeicher zurück geschrieben,
		            wenn die betroffene Cache-Zeile aus dem Cache entfernt
		            wird.\par (LRU-Algorithmus \emph{Last Recently Used})
	      \end{enumerate}

	\item Cache-Verfehlen
	      \begin{enumerate}
		      \item\textbf{Write Allocation} Erzeugen eines neuen
		            Cache-Eintrags, in den die Daten geschrieben werden.
		      \item\textbf{Write Around} Schreiben von Daten direkt in den
		            Hauptspeicher, ohne einen Cache-Eintrag zu erzeugen.
	      \end{enumerate}
\end{enumerate}
% ======================================================================== %
\columnbreak
\section{Sprungvorhersage}
Dynamische Sprungvorhersage mit hilfe einer History-Tabelle mit \(n\)
Entscheidungs-Bits pro Eintrag kann als ein endlicher Automat dargestellt
werden.
% ------------------------------------------------------------------------ %
\subsection{2-Bit-Sprungvorhersage}
\begin{center}
	\includestandalone[width=\linewidth]{./Documents/Graphics/Sprungvorhersage-Abbildung-1}
\end{center}
% ------------------------------------------------------------------------ %
\subsection{3-Bit-Sprungvorhersage}
\begin{center}
	\includestandalone[width=\linewidth]{./Documents/Graphics/Sprungvorhersage-Abbildung-2}
\end{center}
% ======================================================================== %
\section{Cache-Kohärenz}
\begin{description}
	\item[Modified] Cache-Zeile ist gültig und die Kopie im Hauptspeicher
	      ist nicht gültig. Der Eintrag kann in keinem andern Cache
	      existiren.
	\item[Exclusive] Cache-Zeile ist gültig und die Kopie im Hauptspeicher
	      ist gültig. Der Eintrag kann in keinem andern Cache existiren.
	\item[Shared] Cache-Zeile ist gültig und die Kopie im Hauptspeicher
	      ist gültig. Der Eintrag kann möglicherweiße in einem andern Cache
	      existiren.
	\item[Invalid] Cache-Zeile ist nicht gültig oder ist im aktuellen Cache
	      nicht vorhanden.
	\item[Owned] Ist eine Cache-Zeile \textit{Modified} und ein anderer
	      Prozess möchte den aktuellen Wert lesen, so versetzt er die
	      aktuell modifzierte Zeile in den Zustand Owned. Die Caches
	      tauschen dabei ihre Werte aus, jedoch beibt der Wert im
	      Hauptspeicher unverändert und somit ungültig.
	\item[Forward] Die Cache-Zeile agiert in diesem Fall als ein Vermittler
	      für andere Caches in welchem die selbe Cache-Zeile mit dem Zustand
	      \textit{Shared} ist. Dabei ist die Cache-Zeile mit dem Zustand
	      \textit{Forward} immer die welche als letztes modifzierte wurde.
\end{description}
% ------------------------------------------------------------------------ %
\subsection{MSI}
\begin{center}
	\includestandalone[width=0.8\linewidth]{./Documents/Graphics/Cache-Kohaerenz-Abbildung-1}
	{\small\textit{Zustandsübergänge des initiierenden Prozessors}}
	\includestandalone[width=0.8\linewidth]{./Documents/Graphics/Cache-Kohaerenz-Abbildung-2}
	{\small\textit{Zustandsübergänge des passiven Prozessors}}
\end{center}
% ------------------------------------------------------------------------ %
\subsection{MESI}
\begin{center}
	\includestandalone[width=0.8\linewidth]{./Documents/Graphics/Cache-Kohaerenz-Abbildung-3}
	{\small\textit{Zustandsübergänge des initiierenden Prozessors}}
	\includestandalone[width=0.8\linewidth]{./Documents/Graphics/Cache-Kohaerenz-Abbildung-4}
	{\small\textit{Zustandsübergänge des passiven Prozessors}}
\end{center}
% ------------------------------------------------------------------------ %
\subsection{MOESI}
\begin{center}
	\includestandalone[width=0.8\linewidth]{./Documents/Graphics/Cache-Kohaerenz-Abbildung-5}
	{\small\textit{Zustandsübergänge des initiierenden Prozessors}}
	\includestandalone[width=0.8\linewidth]{./Documents/Graphics/Cache-Kohaerenz-Abbildung-6}
	{\small\textit{Zustandsübergänge des passiven Prozessors}}
\end{center}
% ------------------------------------------------------------------------ %
\subsection{MESIF}
\begin{center}
	\includestandalone[width=0.8\linewidth]{./Documents/Graphics/Cache-Kohaerenz-Abbildung-7}
	{\small\textit{Zustandsübergänge des initiierenden Prozessors}}
	\includestandalone[width=0.8\linewidth]{./Documents/Graphics/Cache-Kohaerenz-Abbildung-8}
	{\small\textit{Zustandsübergänge des passiven Prozessors}}
\end{center}
% ======================================================================== %
\columnbreak
\section{Multithreading}
Beispiel Verteilung von 3 Threads auf 4 Ausfüehrungseinheiten, dabei
benötigt die Ausführung einer Instruktion genau eine Taktzyklus:
\par\vspace{\abovedisplayskip}

\textbf{Thread X}\par\vspace{\abovedisplayskip}
\begin{tabularx}{\linewidth}{|c|XXXXXX|}\hline
	X    & 1  & 2  & 3 & 4 & 5 & 6  \\\hline
	AE 1 & X1 & X4 & - & - & - & X6 \\
	AE 2 & X2 & X5 & - & - & - & X7 \\
	AE 3 & X3 &    & - & - & - &    \\
	AE 4 &    &    & - & - & - &    \\\hline
\end{tabularx}\par\vspace{\belowdisplayskip}

\textbf{Thread Y}\par\vspace{\abovedisplayskip}
\begin{tabularx}{\linewidth}{|c|XXXXXX|}\hline
	Y    & 1  & 2  & 3  & 4 & 5 & 6  \\\hline
	AE 1 & Y1 & Y2 & Y4 & - & - & X7 \\
	AE 2 &    & Y3 & Y5 & - & - & X8 \\
	AE 3 &    &    & Y6 & - & - &    \\
	AE 4 &    &    &    & - & - &    \\\hline
\end{tabularx}\par\vspace{\belowdisplayskip}

\textbf{Thread Z}\par\vspace{\abovedisplayskip}
\begin{tabularx}{\linewidth}{|c|XXXXXX|}\hline
	Z    & 1  & 2 & 3  & 4 & 5  & 6 \\\hline
	AE 1 & Z1 & - & Z5 & - & Z7 & - \\
	AE 2 & Z2 & - & Z6 & - &    & - \\
	AE 3 & Z3 & - &    & - &    & - \\
	AE 4 & Z4 & - &    & - &    & - \\\hline
\end{tabularx}\par\vspace{\belowdisplayskip}
% ------------------------------------------------------------------------ %
\subsection{Feinkörniges Multithreading}
Mit Round-Robin Ausführungsreihenfolge wird in jedem Taktzyklus ein anderer
Thread ausgeführt. Falls ein Thread keine Aufgaben hat, wird nicht zurück
gegangen. \par
\begingroup\setlength\tabcolsep{2pt}
\small
\vspace{\abovedisplayskip}
\begin{tabularx}{\linewidth}{|c|XXXXXXXXXX|}\hline
	     & 1  & 2  & 3  & 4  & 5  & 6  & 7  & 8  & 9  & 10 \\\hline
	AE 1 & X1 & Y1 & Z1 & X4 & Y2 & Z5 & Y4 & Z7 & X6 & Y7 \\
	AE 2 & X2 &    & Z2 & X5 & Y3 & Z6 & Y5 &    & X7 & Y8 \\
	AE 3 & X3 &    & Z3 &    &    &    & Y6 &    &    &    \\
	AE 4 &    &    & Z4 &    &    &    &    &    &    &    \\\hline
\end{tabularx}
\endgroup
% ------------------------------------------------------------------------ %
\subsection{Grobkörniges Multithreading}
Führt eine Thread aus, bis eine Verzögerung auftritt, dann wird nach Round-Robin
die Ausführungsreihenfolge geändert. Im Normalfall tritt eine Verzögerung auf,
hier ist es 1 Taktzyklus.\par
\begingroup\setlength\tabcolsep{1.25pt}
\small
\vspace{\abovedisplayskip}
\begin{tabularx}{\linewidth}{|c|XXXXXXXXXXXXXXXX|}\hline
	     & 1  & 2  & 3 & 4  & 5  & 6  & 7 & 8  & 9 & 10 & 11 & 12 & 13 & 14 & 15 & 16 \\\hline
	AE 1 & X1 & X4 & - & Y1 & Y2 & Y4 & - & Z1 & - & X6 & -  & Y7 & -  & Z5 & -  & Z7 \\
	AE 2 & X2 & X4 & - &    & Y3 & Y5 & - & Z2 & - & X7 & -  & Y8 & -  & Z6 & -  &    \\
	AE 3 & X3 &    & - &    &    & Y6 & - & Z3 & - &    & -  &    & -  &    & -  &    \\
	AE 4 &    &    & - &    &    &    & - & Z4 & - &    & -  &    & -  &    & -  &    \\\hline
\end{tabularx}
\endgroup
% ------------------------------------------------------------------------ %
\subsection{Simultanes Multithreading} Mit Round-Robin Ausführungsreihenfolge
werden die Threads aus alle Ausfüehrungseinheiten aufgeteilt. Dabei gibt es eine
Verzögerung falls alle Threads zu einem Taktzyklus keine Aufgaben haben.\par
\begingroup\setlength\tabcolsep{2pt}
\small
\vspace{\abovedisplayskip}
\begin{tabularx}{\linewidth}{|c|XXXXXXXX|}\hline
	     & 1  & 2  & 3  & 4  & 5  & 6 & 7  & 8  \\\hline
	AE 1 & X1 & Y2 & Z3 & Y4 & Z5 &   & X6 & Z7 \\
	AE 2 & X2 & Y3 & Z4 & Y5 & Z6 &   & X7 &    \\
	AE 3 & X3 & Z1 & X4 & Y6 &    &   & Y7 &    \\
	AE 4 & Y1 & Z2 & X5 &    &    &   & Y8 &    \\\hline
\end{tabularx}
\endgroup
\end{multicols*}
\end{document}
