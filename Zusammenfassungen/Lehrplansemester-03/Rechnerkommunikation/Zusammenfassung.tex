\documentclass
[
	8pt,		% font size
	ngerman,	% hyphenation and more
	a4paper,	% paper size
	landscape,	% orientation
	final		% document status (final/draft)
]{extarticle}

% adjust language %
\usepackage[ngerman]{babel}
% integration of speacial characters %
\usepackage[utf8]{inputenc}
\usepackage[T1]{fontenc}
\usepackage{textcomp}
% adjust page layout %
\usepackage{titlesec}
\usepackage{fancyhdr}
% --------------------------- %
\usepackage{multicol}
\usepackage{multirow}
% --------------------------- %
\usepackage{setspace}
\usepackage{geometry}
\usepackage{adjustbox}
% adjust colors %
\usepackage{color}
% integration of mathematical symbols %
\usepackage{amsmath}
\usepackage{amssymb}
\usepackage{amsthm}
\usepackage{mathtools}
% integration of source code %
\usepackage{listings}
% adjust enumerations %
\usepackage{enumitem}
% adjust tables %
\usepackage{tabularx}
% integration of graphics %
\usepackage{graphicx}
% create graphics %
\usepackage{tikz}
\usetikzlibrary{automata, positioning, arrows}
% integration of seperate files %
\usepackage{standalone}

\usepackage{lmodern}
\DeclareMathOperator{\avg}{avg}

% ========== INFORMATION ========== %
\def\name{Louis Seubert}
\def\prefix{Zusammenfassung}
\def\lecture{Rechnersysteme}
\def\maxpages{6}

% ============== ADJUSTMENTS ============== %
% adjust graphics path %
\graphicspath
{
	{figures/}
}
% adjust page layout %
\geometry
{
	left=0.55cm,
	right=0.55cm,
	top=1.10cm,
	bottom=0.55cm,
	headsep=2mm
}
% adjust source code view %
\lstset
{
	basicstyle=\ttfamily\footnotesize,
	columns=fullflexible,
	numbers=left,						% where to put the line-numbers
	numberstyle=\tiny,  				% the style that is used for the line-numbers
	stepnumber=1,
	numbersep=5pt,						% how far the line-numbers are from the code
	showspaces=false,					% show spaces adding particular underscores
	showstringspaces=false,				% underline spaces within strings
	showtabs=false,						% show tabs within strings adding particular underscores
	frame=none,							% adds a frame around the code
	tabsize=2,							% sets default tabsize to 2 spaces
	captionpos=b,						% sets the caption-position to bottom
	breaklines=true,					% sets automatic line breaking
	breakatwhitespace=false,			% sets if automatic breaks should only happen at whitespace
	xleftmargin=10pt					% left margin to prevent number clipping
}

% make header and footer %
\pagestyle{fancy}
\fancyhead{} % clear header
\fancyhead[L]{\prefix\;\lecture}
\fancyhead[R]{\thepage\;--\;\maxpages}
\fancyhead[C]{\name}
\fancyfoot{} % clear footer

% configure document %
\setitemize{leftmargin=15pt}
\setenumerate{leftmargin=15pt}
\setlist{itemsep=1pt,parsep=1pt,noitemsep}

\setlength{\parindent}{0pt}
\setlength{\parskip}{0pt}
\setlength{\topskip}{10pt}
% set column seperator %
\setlength{\columnseprule}{0.5pt}

% change style %
\titleformat*{\section}{\normalsize\bfseries}
\titlespacing*{\section}{0pt}{4pt}{0pt}

\titleformat*{\subsection}{\small\bfseries}
\titlespacing*{\subsection}{0pt}{4pt}{0pt}

\titleformat*{\subsubsection}{\small\bfseries}
\titlespacing*{\subsubsection}{0pt}{4pt}{0pt}

\titleformat*{\paragraph}{\small\bfseries}
\titlespacing{\paragraph}{0pt}{.5em}{.5em}

\titleformat*{\subparagraph}{\footnotesize\bfseries}
\titlespacing*{\subparagraph}{0pt}{.5em}{.5em}

\newcommand*\important{\par\vspace{\abovedisplayskip}\textbf{Wichtig:}\par}
\newcommand*\example{\par\vspace{\abovedisplayskip}\textbf{Beispiel:}\par}
\newcommand{\includefigure}[1]{\begin{center}\input{#1}\end{center}}

\newenvironment{definitions}{
    \par\vspace{\abovedisplayshortskip}\noindent
    \tabularx{\columnwidth}{>{$}l<{$} @{${}={}$} >{\raggedright\arraybackslash}X}
}{\endtabularx\par\vspace{\belowdisplayshortskip}}

\begin{document}
\begin{multicols*}{4}
	%\footnotesize
	\fontsize{6}{6}\selectfont
	% ======================================================================== %
	\section{Protokolle und Stapel}
	% ------------------------------------------------------------------------ %
	\subsection{OSI-Referenzmodell}
	\begin{center}
		\includegraphics[width=\linewidth]{Documents/OSI.pdf}
	\end{center}
	% ------------------------------------------------------------------------ %
	\subsection{IEEE 802}
	\begin{center}
		\includegraphics[width=\linewidth]{Documents/IEEE-802.pdf}
	\end{center}
	\begin{description}
		\item[LLC (Logical Link Control)] 3 Arten von Logical Links:
		      unbestätigt/verbindungslos, bestätigt/verbindungslos,
		      verbindungsorientiert
		\item[MAC (Media Access Control)] Zugriff auf das gemeinsame
		      Medium, z.B.:  CSMA/CD (Ethernet),  CSMA/CA (WLAN)
		\item[PHY] Bitübertragungsschicht
	\end{description}
	% ------------------------------------------------------------------------ %
	\subsection{TCP/IP-Protokollsuite}
	\begin{description}
		\item[Anwendungsschicht (Application Layer)]\;\par
		      FTP, HTTP, SMTP, Telnet, DNS, DHCP
		\item[Transportschicht (Transport Layer)]\;\par
		      TCP, UDP
		\item[Internetschicht (Internet Layer)]\;\par
		      IP, ICMP, ARP, Multicast IP, Mobile IP
		\item[Netzwerkschicht (Network Layer)]\;\par
		      SLIP, PPP, Ethernet, Token Ring, WLAN
	\end{description}
	% ------------------------------------------------------------------------ %
	\subsection{Adressierung}
	\begin{center}
		\includegraphics[width=0.7\linewidth]{Documents/MAC.pdf}
	\end{center}
	% ======================================================================== %
	\section{Bitübertragung}
	% ------------------------------------------------------------------------ %
	\subsection{Betriebsweisen}
	\begin{description}
		\item[Synchron] Zentraler Takt; Explizite Sendefreigabe durch den
		      Empfänger
		\item[Asynchron] Start-Stop-Erkennung notwendig; In der Regel langsamer
		      als synchron
		\item[Simplex] \(S \rightarrow E\)
		\item[Simplex mit Quittung] \(S \overset{\text{Daten}}{\longrightarrow} E \overset{\text{Quittung}}{\longrightarrow} S\)
		\item[Halbduplex] Sender und Empfänger auf beiden Seiten, teilen
		      gemeinsame Leitung
		\item[Vollduplex] Sender und Empfänger auf beiden Seiten mit eigener
		      Leitung (\(2 \cdot \text{Simplex}\))
	\end{description}
	% ------------------------------------------------------------------------ %
	\subsubsection{Modulationsarten}
	Amplitudenmodulation, Frequenzmodulation, Phasenmodulation
	% ------------------------------------------------------------------------ %
	\subsection{Theoretische Obergrenzen für Datenraten}
	\subsubsection{Nyquist-Frequenz}
	Maximale Schrittgeschwindigkeit bei einer Bandbreite \(B\)
	\[V_\text{max} = 2 \cdot B\]
	Maximale Datenrate in \(\frac{Bit}{s}\) bei \(L\) diskreten Stufen,
	ungestört \[D_\text{max} = 2 \cdot B \cdot \log_{2}(L)\]
	\subsubsection{Rauschsignal}
	Umrechnung des \emph{Signal-Rausch-Abstandes} von \(dB\) in
	\(\frac{\text{Signal}}{\text{Noise}}\)
	\[\text{SNR} = \cfrac{\text{Signal}}{\text{Noise}}\]
	\[\text{SNR [dB]} = 10 \cdot \log_{10}\left(\frac{S}{N}\right)\]
	\subsubsection{Shannon-Hartley-Gesetz}
	Maximale Datenrate bei bandbreitenbegrenztem, gestörten Übertragungskanal
	\[D_{\text{max}} = B \cdot \log_2(1+\frac{S}{N})\]
\end{multicols*}
\end{document}
