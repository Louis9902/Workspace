\documentclass[8pt,a4paper,landscape]{article}

\usepackage[utf8]{inputenc}
\usepackage[T1]{fontenc}
\usepackage[ngerman]{babel}
\usepackage[page]{totalcount}

\usepackage{geometry}
\usepackage{titlesec}
\usepackage{fancyhdr}

\usepackage{amsmath}
\usepackage{amssymb}

\usepackage{listings}
\usepackage{enumitem}

\usepackage{tikz}
\usetikzlibrary{automata, positioning, arrows}

\usepackage{graphicx}
\usepackage{tabularx}

\usepackage{multicol}

\DeclareUnicodeCharacter{0308}{~}

% rotate page and set margins %
\geometry{left=0.55cm,right=0.55cm,top=1.10cm,bottom=0.55cm,landscape,headsep=2mm}

% make header and footer %
\pagestyle{fancy}
\fancyhead{} % clear header
\fancyhead[L]{Zusammenfassung Informationssicherheit}
\fancyhead[R]{\thepage\hspace{1mm}von\hspace{1mm}\totalpages} 
\fancyfoot{} % clear footer

% configure document %
\setlist{nosep}
\setlength{\parindent}{0pt}
\setlength{\parskip}{0pt}
\setlength{\topskip}{10pt}
\setlength{\columnseprule}{0.5pt} % column seperator line

% change style %
\titleformat*{\section}
{\large\bfseries} 
\titlespacing*{\section}
{0pt}{4pt}{0pt}

\titleformat*{\subsection}
{\normalsize\bfseries}
\titlespacing*{\subsection}
{0pt}{4pt}{0pt}

\titleformat*{\subsubsection}
{\normalsize\bfseries}
\titlespacing*{\subsubsection}
{0pt}{4pt}{0pt}

\title{Zusammenfassung Informationssicherheit} 
\author{Louis Seubert}

\newcommand{\ord}{\operatorname{ord}}

\usepackage{./msc}

\newcommand{\N}{\ensuremath{\mathbb{N}}}
\newcommand{\No}{\ensuremath{\mathbb{N}_0}}
\newcommand{\Z}[1]{\ensuremath{\mathbb{Z}_{#1}}}
\newcommand{\Zx}[1]{\ensuremath{\mathbb{Z}_{#1}^{*}}}

\newcommand{\calO}{\ensuremath{\mathcal{O}}}

\newcommand{\plaint}{\ensuremath{m}}
\newcommand{\ciphert}{\ensuremath{c}}
\newcommand{\key}{\ensuremath{k}}
\newcommand{\skey}{\ensuremath{sk}}
\newcommand{\pkey}{\ensuremath{pk}}
\newcommand{\secpara}{\ensuremath{k}}
\newcommand{\enc}{\textsc{Enc}}
\newcommand{\dec}{\textsc{Dec}}
\newcommand{\sig}{\textsc{Sig}}
\newcommand{\ver}{\textsc{Ver}}

\newcommand{\complexity}[1]{\par\vspace{\abovedisplayskip}\texttt{Worst-Case-Laufzeit:}\quad#1\par}

\newenvironment{definitions}{
    \par\vspace{\abovedisplayshortskip}\noindent
    \tabularx{\columnwidth}{>{$}l<{$} @{${}={}$} >{\raggedright\arraybackslash}X}
}{\endtabularx\par\vspace{\belowdisplayshortskip}}

\begin{document}
\begin{multicols*}{4}
  \section{Grundlagen}
  \subsection*{Allgemeines}
  \[
   x^{k} \bmod p \equiv \left( x \bmod p \right)^{k \bmod \varphi(p)} \bmod p
  \]
  \[
   \gcd(a,m) = 1 \Rightarrow a^{\varphi(m)} \equiv 1 \bmod m
  \]
  \subsection*{Berechnung der Stellenzahl}
  Die Anzahl $a$ der Ziffern der $b$-adischen Darstellung einer natürlichen Zahl
  $n \in \No$ berechnet sich wie folgt:
  \[
   a =
    \begin{cases}
    1, & \text{wenn } n = 0,\\
    \lfloor \log_b{n} \rfloor + 1, & \text{wenn } n \geq 1.
    \end{cases}
  \]

  \subsection{Phi-Funktion}
  Die Eulersche Phi-Funktion gibt an, wie viele ganze Zahlen teilefremd zu $n$ sind.
  \begin{enumerate}
    \item $\varphi(\text{prime}) = \text{prime} - 1$
    \item $\varphi(\text{prime}^k) = \text{prime}^{k-1} \cdot (\text{prime}-1)$
    \item $\varphi(n \cdot m) = \varphi(n) \cdot \varphi(m)$
  \end{enumerate}
  \[
    \varphi(n) = n \prod_{p \mid n} \left( 1 - \frac{1}{p} \right)
  \]
  Wobei $p$ eine Primzahl ist welche $n$ ganzzahlig teilt. (Primfaktoren)

  \subsection{Kontravalenz}
  \begin{center}
    \begin{tabular}{c| c c }
      $\oplus$ & 0 & 1 \\ \hline
      0        & 0 & 1 \\
      1        & 1 & 0
    \end{tabular}
  \end{center}
  
  \subsection{Diskreter Logarithmus}
  Der diskrete Logarithmus ist die kleinste Lösung für $x$ der Gleichung
  $a^{x} \equiv m \bmod p$ mit $m,\;a \in \N,\;p \in \Z{p}$. \par
  Da sich die diskrete Exponentiation leicht berechnen lässt, gilt das nicht für die
  Umkehrfunktion. \emph{(Diffie-Hellman-Annahme)} Aufgrund dessen wird diese "Einwegfunktion"
  u. a. im Diffie-Hellman-Key-Exchange, der ElGamal-Encryption und vielem mehr eingesetzt.
  Jedoch ist diese Funktion ungeignet für Verschlüsselungsmethoden, da es keine "Falltür" zum
  entschlüssel gibt.
  
  \paragraph{Ordnung} einer Zahl ist der kleinste Exponent, so dass gilt:
  \[
   x^{n} \bmod m \equiv 1
  \]
  Dabei entspricht die Ordnung von $x$ der Anzahl der Elementen welche einen Zyklus bilden.
  Wenn $x$ eine Primzahl ist dann gilt: $\ord(x) = x - 1$
  \[
   \ord(x^{l}) = \frac{\ord(x)}{\gcd(\ord(x),l)}
  \]
  
  \subsection{Modulares Potenzieren}
  Seien $x,k,m \in \N$, gesucht ist $z = x^{k} \mod m$
  \begin{enumerate}
   \item Binärdarstellung von $k$
   \item Ersetzen jeder $0$ durch \textbf{Q} und jeder $1$ durch \textbf{QM}
   \item Dabei wird \textbf{Q} als Anweisung zum \emph{Quadrieren} und \textbf{M} als Anweisung zum
    \emph{Multiplizieren} mit der Basis $x$ aufgefasst.
   \item Begonnen wird mit 1 bzw. kann die erste \textbf{QM} Anweisung durch $x$ substituiert werden.
  \end{enumerate}
  
  \subsection{Chinesischer Restsatz}
  Seien $m_{1}, \ldots, m_{n} \in \N$ paarweise teilerfremd, dann hat das System von Kongruenzen eine
  eindeutige Lösung $x \in \Z{m}$, wobei $m = m_{1} \cdot \ldots \cdot m_{n}$ das Produkt der
  einzelnen Module ist.
  \[
   x \equiv a_{1} \bmod m_{1}, \;\ldots, x \equiv a_{n} \bmod m_{n}
  \]
  Eine Lösung $x$ kann wie folgt ermittelt werden: 
  \[
   x = \left( \sum_{i = 1}^{n} a_{i} \cdot M_{i} \cdot N_{i} \right) \bmod m
  \]
  mit folgenden Vorraussetztungen:
  \begin{definitions}
   $$m$$ & $m = m_{1} \cdot \ldots \cdot m_{n}$ \\
   $$M_{i}$$ & $\frac{m}{m_{i}}$ \\
   $$N_{i}$$ & $M_{i}^{-1} \bmod m_{i}$
  \end{definitions}

  \subsection{Euklidischer Algorithmus}
  Setze $r_{0} := a, r_{1} := b$
  \begin{eqnarray*}
   r_{0} &=& q_2 \cdot r_{1} + r_{2} \\
   r_{1} &=& q_3 \cdot r_{2} + r_{3} \\
   &\vdots& \\
   r_{n-2} &=& q_{n} \cdot r_{n-1} + r_{n} \\
   r_{n-1} &=& q_{n+1} \cdot r_{n} + 0
  \end{eqnarray*}
  \subsubsection*{Erweiterung}
  {\small
  \begin{align*}
   x_{0} &= 1 \quad x_{1} = 0 & y_{0} &= 0 \quad y_{1} = 1 \\
   x_{2} &= x_{0} - q_{2} \cdot x_{1}       & y_{2} &= y_{0} - q_{2} \cdot y_{1} \\
   x_{3} &= x_{1} - q_{3} \cdot x_{2}       & y_{3} &= y_{1} - q_{3} \cdot y_{2} \\
   x_{n} &= x_{n-2} - q_{n} \cdot x_{n-1}   & y_{n} &= y_{n-2} - q_{n} \cdot y_{n-1}
  \end{align*}}
  dann gilt $x_{n}a + y_{n}b = \gcd(a,b)$.
  
  \subsection{Primitivwurzeln}
  Hat die Ordnung einer Zahl $x$ modulo $m$ den größtmöglichen Wert, also $\ord$
  
  \paragraph{Primitivwurzeltest} Um festzustellen, ob eine Zahl $g$ eine Primitivwurzel in der
  Restklassengruppe \Zx{p} mit $p$ ist Primzahl ist, führe man folgende Schritte aus:
  \begin{enumerate}
   \item Primfaktorzerlegung von $p-1$:\par
   $p-1 = p_{1} \cdot \ldots \cdot p_{i}$
   \item Prüfe für alle $q \in \{p_{1}, \;\ldots, p_{i}\}$ ob gilt
   $g^{(p-1)/q} \not\equiv 1 \bmod p$ \par
   Sollten demnach alle Primfaktoren ungleich $1 \bmod p$ sein, dann ist $g$ eine Primitivwurzel.
  \end{enumerate}
  
  Falls $g$ eine Primitivwurzel von \Zx{n} ist, dann ist auch $a = x^{t}$ eine Primitivwurzel
  von \Zx{n} genau dann wenn gilt: $\gcd(t,\varphi(n)) = 1$.
    
 \subsection{Miller-Rabin}
 \vspace{2.27cm}
 
 \section{Verschlüsselungsalgorithmen}
 \subsection{Asymetrische Verfahren}
 \subsubsection{RSA}
 $\skey = (n,d) \quad \pkey = (n,e)$
 
 \paragraph{Schlüsselwahl}
 \begin{enumerate}[itemsep=2pt,leftmargin=15pt]
  \item Wähle Primzahlen $p,q$
  \item $n = p \cdot q$
  \item $\varphi(n) = (p-1) \cdot (q-1)$
  \item Wähle $e$ so dass gilt:\par
  $\gcd(e,\varphi(n)) = 1 = e \cdot d + k \cdot \varphi(n)$
 \end{enumerate}
 %\textit{Anmerkung:} Die Anzahl der möglichen Exponenten lässt sich durch $\varphi(\varphi(n))$ berechnen.
 \paragraph{Verschlüsselung}
 \begin{enumerate}[itemsep=2pt,leftmargin=15pt]
  \item $\ciphert = \plaint^e \bmod n$
 \end{enumerate}
 
 \paragraph{Entschlüsselung}
 \begin{enumerate}[itemsep=2pt,leftmargin=15pt]
  \item $\plaint = \ciphert^d \bmod n$
 \end{enumerate}

 \paragraph{Signieren}
 \begin{enumerate}[itemsep=2pt,leftmargin=15pt]
  \item $s = m^d \bmod n$
 \end{enumerate}

 \paragraph{Verifizieren}
 \begin{enumerate}[itemsep=2pt,leftmargin=15pt]
  \item $m = s^e \bmod n$
 \end{enumerate}
 
 \subsubsection{ElGamal}
 $\skey = (p,g,x) \quad \pkey = (p,g,y)$
 
 \paragraph{Schlüsselwahl}
 \begin{enumerate}
  \item Wähle Primzahl $p$ sowie $g$ welches eine Primitivwurzel von $\Z{p}$ ist.
  \item Wähle eine zufällige ganze Zahl $x \in \{1,\;\ldots,\;p-2\}$
  \item Berechne: \par
  $y = g^x \bmod p$
 \end{enumerate}
 
 \paragraph{Verschlüsselung}
 \begin{enumerate}
  \item Wähle $k \in [1;p-2]$
  \item $c_1 = g^k \bmod p$
  \item $c_2 = m \cdot y^k \bmod p$
  \item $c = (c_1,c_2)$
 \end{enumerate}
 
 \paragraph{Entschlüsselung}
 \begin{enumerate}
  \item $m = c_{1}^{p-1-x} \cdot c_{2} \mod p$
 \end{enumerate}
 
 \subsection{Symmetrische Verfahren}
 \subsubsection{DES}
 \begin{center}
  \begin{tabular*}{.9\linewidth}{r|l@{\extracolsep{\fill}}}\hline
   Struktur & Feistelchiffre    \\\hline
   Schlüssellänge & 56 Bit      \\\hline
   Blocklänge & 64 Bit          \\\hline
   Rundenanzahl & 16            \\\hline
 \end{tabular*}
 \end{center}

 \subsubsection{AES}
  \begin{center}
  \begin{tabular*}{.9\linewidth}{r|l@{\extracolsep{\fill}}}\hline
   Struktur & Substitutionschiffre          \\\hline
   Schlüssellänge & 128, 192 oder 256 Bit   \\\hline
   Blocklänge & 128 Bit                     \\\hline
   Rundenanzahl & 10, 12 oder 14            \\\hline
 \end{tabular*}
 \end{center}
 
 \subsection{Blockverschlüsselung}
 \begin{center}
  \begin{tabular}{c | c}
      \textbf{Blockorientiert} & \textbf{Stromorientiert} \\\hline
      ECB & OFB \\
      CBC & CFB
  \end{tabular}
 \end{center}
 Stromorientierte Betriebsarten erfordern kein Padding das Klartextes, und
 unterstützen keine asymetrische Verschlüsselung. \par
 Um einen Klartext \plaint verschlüsseln zu können muss diser in Blöcke der 
 Länge $r \leq n$ eingeteilt werden. Dabei ist $n$ die Blocklänge des 
 Verschlüsselungsverfahren, $r$ die Länge der Klartextblöcke. $\plaint_i$
 bezeichnet dabei einen Block des Klartextes \plaint.
 
 \subsubsection{ECB}
 \subsubsection{CBC}
 \subsubsection{CFB}
 \subsubsection{OFB}
 
 \section{Hashfunktionen}
 Eine Hashfunktion ist eine Einwegfunktion welche einen belibig langen Bitstrom
 auf einen Bitstrom mit fester länge minimiert.
 Eine Hashfunktion kann folgende Eigenschaften besitzen:
 \begin{description}
  \item[Kollisionsresistenz]
  \item[Urbildresistenz]
 \end{description}
 
 \subsection{Message Authentification Codes}
 \subsubsection{Hash Message Authentication Code}

 
 \section{Protokolle}
 \subsection{Diffie-Hellman-Schlüsselaustausch}
 Alice und Bob haben einen gemeinsamen öffentlichen Schlüssel $(p,g)$, wobei
 $p$ eine Primzahl ist und $g$ eine Primitivwurzel von \Z{p}.
 \begin{enumerate}
  \item Alice wählt zufälliges $a \in [0;p-2]$ und berechnet $c = g^a \bmod p$
  und übermittelt $c$ an Bob.
  \item Bob wählt zufälliges $b \in [0;p-2]$ und berechnet $d = g^b \bmod p$
  und übermittelt $d$ an Alice.
  \item Alice berechnet nun $k = d^a \bmod p$
  \item Bob berechnet nun $k = c^b \bmod p$
 \end{enumerate}
 
 \subsection{Station-to-Station-Protokoll}
 Eine Erweiterung des Diffie-Hellman-Schlüsselaustausch um einen 
 Man-in-the-Middle-Angriff auszuschließen.
 \begin{enumerate}
  \item Alice wählt zufälliges $a \in [0;p-2]$ und berechnet $c = g^{a} \bmod p$
  und übermittelt $c$ an Bob.
  \item Bob wählt zufälliges $b \in [0;p-2]$ und berechnet $k = g^{ab} \bmod p$
  \item Bob sendet nun $g^{b}$ sowie $z = \enc_{k}(s)$ mit $s = \sig_{\skey_{B}}(g^{a}||g^{b})$
  \item Alice $k = g^{ab} \bmod p$ und $s = \dec_{k}(z)$ sowie $\ver((g^{a}||g^{b}),s,\pkey_{B})$
  \item Alice sendet nun $z = \enc_{k}(s)$ mit $s = \sig_{\skey_{A}}(g^{b}||g^{a})$
  \item Bob entschlüsselt $s = \dec_{k}(z)$ und verifiziert $\ver((g^{b}||g^{a}),s,\pkey_{A})$
 \end{enumerate}
 Sollte die letzte Überprüfung korrekt sein, dann ist ein gemeinsamer Schlüssel gewählt.
  
\end{multicols*}
\end{document}
