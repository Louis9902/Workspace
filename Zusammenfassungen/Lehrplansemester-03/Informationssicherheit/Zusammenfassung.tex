\documentclass[a4paper,landscape]{article}

\usepackage[utf8]{inputenc}
\usepackage[T1]{fontenc}
\usepackage[ngerman]{babel}

\usepackage{geometry}
\usepackage{titlesec}
\usepackage{fancyhdr}

\usepackage{amsmath}
\usepackage{amssymb}

\usepackage{listings}
\usepackage{enumitem}

\usepackage{graphicx}
\usepackage{tabularx}

\usepackage{multicol}

\usepackage{standalone}

\usepackage{tikz}
\usetikzlibrary{calc,arrows,positioning}

\hyphenation{
	prak-tisch
	un-mög-lich
	Schlü-sel
	Schlüs-sel-aus-tausch
	Diffie-Hell-man-Schlüs-sel-aus-tausch
}

% define some static variables %
\def\creator{Louis Seubert}
\def\subject{Informationssicherheit}
\def\maxpage{4}

% rotate page and set margins %
\geometry{left=0.55cm,right=0.55cm,top=1.10cm,bottom=0.55cm,landscape,headsep=2mm}

% make header and footer %
\pagestyle{fancy}
\fancyhead{} % clear header
\fancyhead[L]{Zusammenfassung \subject}
\fancyhead[R]{\thepage\;von\;\maxpage}
\fancyhead[C]{\creator}
\fancyfoot{} % clear footer

% configure document %
\setlist{itemsep=1pt,parsep=0pt}
\setlength{\parindent}{0pt}
\setlength{\parskip}{0pt}
\setlength{\topskip}{10pt}
\setlength{\columnseprule}{0.5pt} % column seperator line

% change style %
\titleformat*{\section}{\large\bfseries}
\titlespacing*{\section}{0pt}{4pt}{0pt}

\titleformat*{\subsection}{\normalsize\bfseries}
\titlespacing*{\subsection}{0pt}{4pt}{0pt}

\titleformat*{\subsubsection}{\normalsize\bfseries}
\titlespacing*{\subsubsection}{0pt}{4pt}{0pt}

\titleformat*{\paragraph}{\normalsize\bfseries}
\titlespacing{\paragraph}{0pt}{.5em}{.5em}

\titleformat*{\subparagraph}{\small\bfseries}
\titlespacing*{\subparagraph}{0pt}{.5em}{.5em}

\title{Zusammenfassung \subject} 
\author{Louis Seubert}

% define some macros %
\newcommand{\ord}{\operatorname{ord}}

\newcommand{\plaint}{\ensuremath{m}}
\newcommand{\ciphert}{\ensuremath{c}}

\newcommand{\skey}{\ensuremath{sk}}
\newcommand{\pkey}{\ensuremath{pk}}

\newcommand{\enc}{\textsc{Enc}}
\newcommand{\dec}{\textsc{Dec}}
\newcommand{\sig}{\textsc{Sig}}
\newcommand{\ver}{\textsc{Ver}}

% define some enviroments %

\newenvironment{definitions}{
	\par\vspace{\abovedisplayshortskip}\noindent
	\tabularx{\columnwidth}{>{$}l<{$} @{${}={}$} >{\raggedright\arraybackslash}X}
}{\endtabularx\par\vspace{\belowdisplayshortskip}}

\begin{document}
\begin{multicols*}{4}
	\section{Grundlagen}
	\subsection*{Allgemeines}
	\[x^{k} \bmod p \equiv \left( x \bmod p \right)^{k \bmod \varphi(p)} \bmod p\]
	\[\gcd(a,m) = 1 \Rightarrow a^{\varphi(m)} \equiv 1 \bmod m\]

	\subsection*{Berechnung der Stellenzahl}
	Die Anzahl \(a\) der Ziffern der \(b\)-adischen Darstellung einer natürlichen
	Zahl \(n \in \mathbb{N}_{0}\) berechnet sich wie folgt:
	\[
		a =
		\begin{cases}
			1,                               & \quad \text{wenn } n = 0    \\
			\lfloor \log_{b}{n} \rfloor + 1, & \quad \text{wenn } n \geq 1
		\end{cases}
	\]

	\subsection*{Teilbarkeit}
	Zwei Zahlen \(a,b \in \mathbb{Z}\) werden als teilerfremd bezeichnet, wenn
	\(\gcd(a,b) = 1\) ist.

	\subsection*{Ordnung}
	Die Ordnung eines Gruppenelementes \(g\) einer Gruppe \((G,\cdot)\) ist die
	kleinste natürliche Zahl \(n > 0\) für die gilt \(g^{n} = e\), wobei \(e\)
	das neutrale Element der Gruppe ist.
	\paragraph*{Multiplikative Ordnung} \(\ord_{m}(a)\) ist die multiplikative
	Ordnung modulo \(m\) des Elementes \(a\), d.h. der kleinste positive Exponent
	\(n\) für den gilt: \[x^{n} \equiv 1 \pmod{m}\] Eine Erweiterung dessen ist:
	\[\ord(x^{l}) = \frac{\ord(x)}{\gcd(\ord(x),l)}\]
	\paragraph*{Primzahlen} Die Ordnung für Primzahlen lässt sich wie folgt
	bestimmen: \[\ord(p) = \varphi(p) = p -1 \]

	\subsection{Multiplikative Inverse}
	Das Multiplikative Inverse von \(a\) im Modul \(m\) lässt sicht mit dem
	erweiterten euklidischen Algorithmus berechnen. Der Algorithmus liefert die
	Linearkombination
	\[\gcd(a,m) = u \cdot a + v \cdot m = 1\]
	wenn \(a\) und \(m\) teilerfremd sind. Somit lässt sich das Inverse dann
	einfach ablesen:
	\[a^{-1} \equiv u \bmod m\]

	\subsection{Eulersche Phi-Funktion}
	Die Eulersche Phi-Funktion gibt an, wie viele ganze Zahlen teilefremd zu
	\(n\) sind. Wenn \(p\) eine Primzahl ist dann kann man folgende aussagen
	treffen:
	\[\varphi(p) = p - 1\]
	\[\varphi(p^{k}) = p^{k-1} \cdot \left(p - 1\right)\]
	\[\varphi(n \cdot m) = \varphi(n) \cdot \varphi(m)\]
	\[\varphi(n) = n \prod_{p|n} \left(1 - \frac{1}{p}\right)\]
	Wobei \(p|n\), die Primfaktoren der Zahl \(n\) sind.


	\subsection{Kontravalenz}
	\begin{center}
		\begin{tabular}{c| c c }
			$\oplus$ & 0 & 1 \\ \hline
			0        & 0 & 1 \\
			1        & 1 & 0
		\end{tabular}
	\end{center}

	\subsection{Diskreter Logarithmus}
	Der diskrete Logarithmus ist die kleinste Lösung für \(x\) der Gleichung
	\(a^{x} \equiv m \bmod p$ mit $m,\;a \in\mathbb{N},\;p \in\mathbb{Z}_{p}\). \par
	Da sich die diskrete Exponentiation leicht berechnen lässt, gilt das nicht
	für die Umkehrfunktion. \emph{(Diffie-Hellman-Annahme)} Aufgrund dessen wird
	diese "Einwegfunktion" u. a. im Diffie-Hellman-Key-Exchange,
	der ElGamal-Encryption und vielem mehr eingesetzt. Jedoch ist diese Funktion
	ungeignet für Verschlüsselungsmethoden, da es keine "Falltür" zum
	entschlüssel gibt.

	\subsection{Modulares Potenzieren}
	Seien \(x,k,m \in\mathbb{N}\), gesucht ist \(z = x^{k} \bmod m\)
	\begin{enumerate}
		\item Binärdarstellung von \(k\)
		\item Ersetzen jeder \(0\) durch \textbf{Q} und jeder \(1\) durch
		      \textbf{QM}
		\item Dabei wird \textbf{Q} als Anweisung zum \emph{Quadrieren} und
		      \textbf{M} als Anweisung zum \emph{Multiplizieren} mit der Basis \(x\)
		      aufgefasst.
		\item Begonnen wird mit 1 bzw. kann die erste \textbf{QM} Anweisung
		      durch \(x\) substituiert werden.
	\end{enumerate}

	\subsection{Chinesischer Restsatz}
	Seien \(m_{1}, \ldots, m_{n} \in\mathbb{N}\) paarweise teilerfremd, dann hat
	das System von Kongruenzen eine eindeutige Lösung \(x \in \mathbb{Z}_{m}\),
	wobei \(m = m_{1} \cdot \ldots \cdot m_{n}\) das Produkt der einzelnen Module
	ist. \[x \equiv a_{1} \bmod m_{1}, \;\ldots, x \equiv a_{n} \bmod m_{n}\]
	Eine Lösung $x$ kann wie folgt ermittelt werden:
	\[x = \left( \sum_{i = 1}^{n} a_{i} \cdot M_{i} \cdot N_{i} \right) \bmod m\]
	mit folgenden Vorraussetztungen:
	\begin{definitions}
		$$m$$ & $m = m_{1} \cdot \ldots \cdot m_{n}$ \\
		$$M_{i}$$ & $\frac{m}{m_{i}}$ \\
		$$N_{i}$$ & $M_{i}^{-1} \bmod m_{i}$
	\end{definitions}

	\subsection{Euklidischer Algorithmus}
	Setze $r_{0} := a, r_{1} := b$
	\begin{eqnarray*}
		r_{0} &=& q_2 \cdot r_{1} + r_{2} \\
		r_{1} &=& q_3 \cdot r_{2} + r_{3} \\
		&\vdots& \\
		r_{n-2} &=& q_{n} \cdot r_{n-1} + r_{n} \\
		r_{n-1} &=& q_{n+1} \cdot r_{n} + 0
	\end{eqnarray*}
	\paragraph{Erweiterung}
	{\small
		\begin{align*}
			x_{0} & = 1 \quad x_{1} = 0             & y_{0} & = 0 \quad y_{1} = 1             \\
			x_{2} & = x_{0} - q_{2} \cdot x_{1}     & y_{2} & = y_{0} - q_{2} \cdot y_{1}     \\
			x_{3} & = x_{1} - q_{3} \cdot x_{2}     & y_{3} & = y_{1} - q_{3} \cdot y_{2}     \\
			x_{n} & = x_{n-2} - q_{n} \cdot x_{n-1} & y_{n} & = y_{n-2} - q_{n} \cdot y_{n-1}
		\end{align*}}
	dann gilt $x_{n}a + y_{n}b = \gcd(a,b)$.

	\subsection{Primitivwurzeln}
	Eine ganze Zahl \(a\) ist eine Primitivwurzel modulo \(m\) wenn gilt dass
	die Ordnung von \(a\) modulo \(m\) gleich der Gruppenordnung der primen
	Restklassengruppe ist: \[\ord_m(a)=\varphi(m)\]

	\paragraph{Primitivwurzeltest} Um festzustellen, ob eine Zahl \(g\) eine
	Primitivwurzel in der Restklassengruppe \(\mathbb{Z}_{p}^{*}\) mit \(p\) ist
	Primzahl ist, führe man folgende Schritte aus:
	\begin{enumerate}
		\item Primfaktorzerlegung von \(p-1\):\par
		      \(p-1 = p_{1} \cdot \ldots \cdot p_{i}\)
		\item Prüfe für alle \(q \in \{p_{1},\ldots, p_{i}\}\) ob gilt
		      \(g^{(p-1)/q} \not\equiv 1 \pmod{p}\) \par
		\item Sollten demnach alle Primfaktoren ungleich \(1 \bmod p\) sein, dann
		      ist \(g\) eine Primitivwurzel.
	\end{enumerate}

	Falls \(g\) eine Primitivwurzel von \(\mathbb{Z}_{p}^{*}\) ist, dann ist auch
	\(a = g^{t}\) eine Primitivwurzel von \(\mathbb{Z}_{p}^{*}\) genau dann wenn
	gilt: \(\gcd(t,\varphi(p)) = 1\). Somit lässt sich folgendes aussagen:
	\[\gcd(t,\varphi(p)) = 1 \Rightarrow \langle g^{t} \rangle = \mathbb{Z}_{p}^{*}\]

	\section{Verschlüsselungsalgorithmen}
	\subsection{Asymetrische Verfahren}
	\subsubsection{RSA}
	\paragraph{Schlüsselerzeugung}
	\begin{enumerate}
		\item Wähle zwei große Primzahlen \(p, q\) mit \(p \neq q\) und
		      vorgegebener Bitlänge \(k\).
		\item Berechne \(n = p \cdot q\).
		\item Berechne \(\varphi(n) = (p - 1)(q - 1)\)
		\item Wähle \(e \in \{3, \dotsc, \varphi(n) - 1\}\), wobei
		      \(\gcd(e, \varphi(n)) = 1\). \\
		      Hier macht es sinn eine Zahl \(e\) zuwählen welche in der
		      Binärdarstellung viele \(0\) hat, da dass Modulare Potenzieren so
		      weniger Multiplikationen besitzt.
		\item Berechne mit Hilfe des erweiterten Euklid das zu \(e\)
		      multiplikativ-inverse Element \(d\) bezüglich
		      \(\varphi(n)\): \par
		      \(\gcd(e,\varphi(n)) = 1 = e \cdot d + k \cdot \varphi(n)\)
		\item \((\pkey, \skey) \leftarrow ((n,e), (n,d))\).
	\end{enumerate}

	\paragraph{Verfahren}
	\begin{align*}
		\enc(\pkey,\plaint)        & = \plaint^{e} \bmod n                               \\
		\dec(\skey,\ciphert)       & = \ciphert^{d} \bmod n                              \\
		\sig(\skey,\plaint)        & = \plaint^{d} \bmod n                               \\
		\ver(\pkey,\plaint,\sigma) & = 1 :\;\Leftrightarrow \plaint = \sigma^{e} \bmod n
	\end{align*}

	\paragraph{Verschlüsselung mit RSA-OAEP}
	Die RSA-OAEP (\emph{optimal asymmetric encryption padding}) Variante ist
	gegen deterministische Angriffe sicher! Die Funktionen \(g(x)\) und \(h(x))\)
	sind Hashfunktionen.

	\subparagraph{Verschlüsselung}
	\begin{enumerate}
		\item Wähle \(r\) zufällig
		\item Berechne \par
		      \(x = m \oplus g(r)\) und \(y = r \oplus h(x)\)
		\item Verschlüssele \par
		      \(\enc_{\text{OAEP}}(\pkey,\plaint) = (x||y)^{e} \bmod n\)
	\end{enumerate}

	\subparagraph{Entschlüsselung}
	\begin{enumerate}
		\item Rekonstruiere \((x||y) = \dec(\skey,\ciphert)\)
		\item Rekonstruiere \(r\) mit \(r = y \oplus h(x)\)
		\item Berechne \(\plaint\) mit \(\plaint = x \oplus g(r)\)
	\end{enumerate}

	\paragraph{Signieren und Entschlüsseln mit dem Chinesischen Restsatz}
	\[
		i =
		\begin{cases}
			c & \text{für Entschlüsseln} \\
			m & \text{für Signieren}
		\end{cases}
	\]
	\[
		o =
		\begin{cases}
			m & \text{für Entschlüsseln} \\
			s & \text{für Signieren}
		\end{cases}
	\]
	\begin{enumerate}
		\item $d_{p} = d \bmod (p-1) \;,\; d_{q} = d \bmod (q-1)$
		\item $i_{1} = i^{d_{p}} \bmod p \;,\; i_{2} = i^{d_{q}} \bmod q$
		\item \(
		      h =
		      \begin{cases}
			      q^{-1} \cdot (i_{1} - i_{2})     & \text{falls } i_{1} > i_{2} \\
			      q^{-1} \cdot (i_{1} - i_{2} + p) & \text{falls } i_{1} < i_{2}
		      \end{cases}
		      \)
		      \
		\item $h = h \bmod p$
		\item $o = i_{2} + (h \cdot q)$
	\end{enumerate}

	\subsubsection{ElGamal}
	\paragraph{Schlüsselerzeugung}
	\begin{enumerate}
		\item Wähle eine große Primzahlen \(p\) mit vorgegebener Bitlänge \(k\)
		\item Suche eine Primitivwurzel \(g\) in der Gruppe \(\mathbb{Z}_{p}\)
		\item Wähle einen zufälligen Exponenten \(x \in\{2,\ldots,\;p-2\}\)
		\item Berechne \(y = g^{x} \bmod p\)
		\item \((\pkey,\skey) \leftarrow ((p,g,y),(p,g,x))\)
	\end{enumerate}

	\paragraph{Verfahren} \,\\
	\(k \in\{1,\ldots,p-2\}\) wird für jedes Verfahren erneut zufällig gewählt.
	\begin{align*}
		\enc(\pkey,\plaint)          & = \left((g^{k}),(y^{k} \cdot \plaint)\right) \pmod{p} \\
		\dec(\skey,(g^{k},\ciphert)) & = (g^{k})^{p-1-x} \cdot \ciphert \pmod{p}             \\
		\sig(\skey,\plaint)          & = (s_{1},s_{2})                                       \\
		\ver(\pkey,\plaint,\sigma)   & = 1 :\;\Leftrightarrow v_{1} = v_{2}
	\end{align*}

	Für \sig\ gilt zudem dass \(\gcd(k,p-1) = 1\)
	\begin{align*}
		s_{1} & := g^{k} \bmod p                                  \\
		s_{2} & := k^{-1} \cdot (\plaint - a \cdot x) \bmod (p-1)
	\end{align*}

	Für \ver\ gilt zudem dass \(1 \leq s_{1} \leq p-1\)
	\begin{align*}
		v_{1} & := g^{\plaint} \bmod p                   \\
		v_{2} & := y^{s_{1}} \cdot s_{1}^{s_{2}} \bmod p
	\end{align*}

	\subsection{Symmetrische Verfahren}
	\subsubsection{DES}
	\begin{center}
		\begin{tabular*}{.9\linewidth}{r|l@{\extracolsep{\fill}}}\hline
			Struktur & Feistelchiffre    \\\hline
			Schlüssellänge & 56 Bit      \\\hline
			Blocklänge & 64 Bit          \\\hline
			Rundenanzahl & 16            \\\hline
		\end{tabular*}
	\end{center}

	\subsubsection{AES}
	\begin{center}
		\begin{tabular*}{.9\linewidth}{r|l@{\extracolsep{\fill}}}\hline
			Struktur & Substitutionschiffre          \\\hline
			Schlüssellänge & 128, 192 oder 256 Bit   \\\hline
			Blocklänge & 128 Bit                     \\\hline
			Rundenanzahl & 10, 12 oder 14            \\\hline
		\end{tabular*}
	\end{center}

	\subsection{Blockverschlüsselung}
	\begin{center}
		\begin{tabular}{c | c}
			\textbf{Blockorientiert} & \textbf{Stromorientiert} \\\hline
			ECB                      & CFB                      \\
			CBC                      & OFB
		\end{tabular}
	\end{center}
	Stromorientierte Betriebsarten erfordern kein Padding das Klartextes, und
	unterstützen keine asymetrische Verschlüsselung. \par
	Um einen Klartext \plaint verschlüsseln zu können muss diser in Blöcke der
	Länge $r \leq n$ eingeteilt werden. Dabei ist $n$ die Blocklänge des
	Verschlüsselungsverfahren, $r$ die Länge der Klartextblöcke. $\plaint_i$
	bezeichnet dabei einen Block des Klartextes \plaint.

	\begin{definitions}
		$\(n\)$ & Die Blocklänge des Verschlüsselungsverfahren \\
		$\(r\)$ & Die Blocklänge der Nachrichtenfragmente
	\end{definitions}

	\subsubsection{ECB}
	\begin{align*}
		r            & = n                      \\
		\ciphert_{i} & = \enc_{k}(\plaint_{i})  \\
		\plaint_{i}  & = \dec_{k}(\ciphert_{i})
	\end{align*}

	\textbf{Übertragungsfehler}
	\begin{description}
		\item[Bitfehler] in \(\ciphert_{i} \Rightarrow \plaint_{i}\) ist zufällig,
		      alle anderen werden korrekt entschlüsselt.
		\item[Verlust] von \(\ciphert_{i} \Rightarrow \plaint_{i}\) ist verloren,
		      alle anderen werden korrekt entschlüsselt.
		\item[Angriff] ist möglich da gleiche verschlüsselte Blöcke die selbe
		      Nachricht enthalten.
	\end{description}

	\subsubsection{CBC}
	\begin{align*}
		r            & = n                                            \\
		\ciphert_{0} & = \text{Initialisierungsvektor}                \\
		\ciphert_{i} & = \enc_{k}(\plaint_{i} \oplus \ciphert_{i-1})  \\
		\plaint_{i}  & = \dec_{k}(\ciphert_{i}) \oplus \ciphert_{i-1}
	\end{align*}

	\textbf{Übertragungsfehler}
	\begin{description}
		\item[Bitfehler] in ... \\
		      ... \(\ciphert_{i} \Rightarrow \plaint_{i}\) ist zufällig
		      und \(\plaint_{i+1}\) hat den Bitfehler an gleicher Stelle wie
		      \(\ciphert_{i}\), alle anderen werden korrekt entschlüsselt. \\
		      ... Initialisierungsvektor \(\Rightarrow \plaint_{1}\) hat den
		      Bitfehler an gleicher Stelle wie der Initialisierungsvektor.
		\item[Verlust] von \(\ciphert_{i} \Rightarrow \plaint_{i}\) ist verloren
		      und \(\plaint_{m+1}\) ist zufällig, \(\plaint_{i+2}\) und folgende werden
		      korrekt entschlüsselt.
		\item[Angriff] ist nicht möglich.
	\end{description}

	\subsubsection{CFB}
	\begin{align*}
		r            & \geq 1                                         \\
		n            & \geq r                                         \\
		x_{0}        & = \text{Initialisierungsvektor}                \\
		x_{i+1}      & = lsb_{n-r}(x_{i}) || \ciphert_{i}             \\
		\ciphert_{0} & = x_{0}                                        \\
		\ciphert_{i} & = \plaint_{i} \oplus msb_{r}(\enc_{k}(x_{i}))  \\
		\plaint_{i}  & = \ciphert_{i} \oplus msb_{r}(\dec_{k}(x_{i}))
	\end{align*}

	\textbf{Übertragungsfehler}
	\begin{description}
		\item[Bitfehler] in ... \\
		      ... \(\ciphert_{i} \Rightarrow \plaint_{i}\) hat den Bitfehler an gleicher
		      Stelle wie \(\ciphert_{i}\), und alle folgenden \(\lceil\frac{n}{r}\rceil\)
		      Blöcke sind zufällig. \\
		      ... Initialisierungsvektor \(\Rightarrow\) die ersten
		      \(\left\lceil\frac{\text{Bitfehlerstelle}}{r}\right\rceil\) Blöcke sind
		      zufällig.
		\item[Verlust] von \(\ciphert_{i} \Rightarrow \plaint_{i}\) ist verloren,
		      und alle folgenden \(\lceil\frac{n}{r}\rceil\) Blöcke sind zufällig.
		\item[Angriff] ist nicht möglich.
	\end{description}

	\subsubsection{OFB}
	\begin{align*}
		r            & \geq 1                               \\
		n            & \geq r                               \\
		x_{0}        & = \text{Initialisierungsvektor}      \\
		x_{i+1}      & = \enc_{k}(x_{i})                    \\
		\ciphert_{0} & = x_{0}                              \\
		\ciphert_{i} & = \plaint_{i} \oplus msb_{r}(x_{i})  \\
		\plaint_{i}  & = \ciphert_{i} \oplus msb_{r}(x_{i})
	\end{align*}

	\textbf{Übertragungsfehler}
	\begin{description}
		\item[Bitfehler] in ... \\
		      \(\ciphert_{i} \Rightarrow \plaint_{i}\) hat den
		      Bitfehler an gleicher Stelle wie \(\ciphert_{i}\). \\
		      ... Initialisierungsvektor \(\Rightarrow\) alles zufällig.
		\item[Verlust] von \(\ciphert_{i} \Rightarrow \plaint_{i}\) ist verloren,
		      und alle folgenden Blöcke sind zufällig (korrigierbar).
		\item[Angriff] ist nicht möglich.
	\end{description}

	\subsubsection{CTR}
	\begin{align*}
		r            & \geq 1                                  \\
		n            & \geq r                                  \\
		ctr_{i}      & =  \text{Initialisierungsvektor}        \\
		\ciphert_{i} & = \plaint_{i} \oplus \enc_{k}(ctr_{i})  \\
		\plaint_{i}  & = \ciphert_{i} \oplus \enc_{k}(ctr_{i})
	\end{align*}

	\textbf{Übertragungsfehler}
	\begin{description}
		\item[Bitfehler] in \(\ciphert_{i} \Rightarrow \plaint_{i}\) hat den
		      Bitfehler an gleicher Stelle wie \(\ciphert_{i}\).
		\item[Verlust] von \(\ciphert_{i} \Rightarrow \plaint_{i}\) ist verloren,
		      und alle folgenden Blöcke sind korrekt.
		\item[Angriff] ist nicht möglich.
	\end{description}
	\underline{Vorteile zum \emph{OFB}:} Wahlfreien Zugriff auf jeden
	verschlüsselten Block und \enc\ und \dec\ können parallel durchgeführt
	werden.

	\section{Angriffe gegen Verschlüsselungsalgorithmen}
	\subsection{RSA}
	\subsubsection*{Common-Modulus-Attack}
	Möglich wenn die selbe Nachricht mit zwei unterschiedlichen Exponenten verschlüsselt wird,
	jedoch die Primfaktoren identisch sind. Die beiden Exponent sind dabei jedoch teilerfremd
	zu einander sind.
	\begin{center}
		\begin{tabular}[t]{ccc}
			\pkey        & \ciphert                    & \plaint \\\hline
			$(n, e_{1})$ & $c_{1} = m^{e_{1}} \bmod n$ & $m$     \\
			$(n, e_{2})$ & $c_{2} = m^{e_{2}} \bmod n$ & $m$     \\\hline
		\end{tabular}
	\end{center}
	\begin{enumerate}
		\item Berechne mit dem erweitertem Euklid $a \cdot e_{1} + b \cdot e_{2} = 1$. \par
		      $a$ oder $b$ wird negativ sein.
		\item Berechne nun folgendes:
		      \begin{align*}
			      m & = m^{1}                                                                                           \\
			        & = m^{a \cdot e_{1} + b \cdot e_{2}} = \left(m^{e_{1}}\right)^{a} \cdot \left(m^{e_{2}}\right)^{b} \\
			        & = \left(c_{1}\right)^{a} \cdot \left(c_{2}\right)^{b} \bmod n
		      \end{align*}
		\item Der negative Exponent kann nun auch wie folgt geschrieben werden: \par
		      $x^{-2} = \left(x^{-1}\right)^{2}$ wobei $x^{-1}$ dem Inversen entspricht.
		\item Sollte es kein Inverses geben (d.h. $\gcd(c_{1|2},m) \neq 1$) dann lassen
		      sich die Primfaktoren wie folg berechnen: \par
		      \(
		      n = \underbrace{\gcd(c_{(1|2)},m)}_{p} \cdot \underbrace{\frac{n}{p}}_{q}
		      \)
	\end{enumerate}

	\subsubsection*{Low-Encryption-Exponent-Attack}
	Möglich bei einem kleinen Abstand zwischen den Primfaktoren von \(n\).
	\begin{enumerate}
		\item \(x = \left\lceil\;\sqrt{n}\;\right\rceil\)
		\item \(y = \sqrt{x^{2}-n}\) (wenn von 3. dann \(x+1\))
		\item \(
		      \begin{cases}
			      \text{gehe zu 2} & \text{wenn } y \textbf{ nicht}\text{ ganzzahlig} \\
			      \text{gehe zu 4} & \text{wenn } y \text{ ganzzahlig}
		      \end{cases}\)
		\item \(p = x+y, q = x-y\)
	\end{enumerate}

	\subsubsection*{Small-Message-Space-Attack}
	Wenn die Anzahl der möglichen Nachrichten klein ist und der mögliche Inhalt
	im Vorraus bekannt ist, dann kann der Angreifer eine Liste führen, in welcher
	die möglichen Nachrichten zusammen mit dem \pkey\; verschlüsselten Nachricht
	aufgeführt werden. Sollte dann eine Nachricht abgefangen werden, muss lediglich
	in der Liste gesucht werden. \par
	Durch den Einsatz von OAEP (Optimal Asymmetric Enrcyption Padding) kann dies
	verhindert werden.

	\subsection{ElGamal}
	\subsubsection*{Berechnungen}
	Ermitteln von \(k\), wenn es mehrmals benutzt wurde um eine Nachricht zu
	signieren:
	\[k = (s_{2_1} - s_{2_2})^{-1} \cdot (\plaint_{1} - \plaint_{2}) \bmod (p-1)\]
	Bei bekannten \(k\) das \(x\) berechnen (Funktioniert nur, wenn das Inverse
	existiert):
	\[x = s_{1}^{-1} \cdot (\plaint - \skey) \bmod (p-1)\]

	\subsubsection*{Existenzielles Fälschen}
	\begin{enumerate}
		\item Wähle \(c\) zufällig mit \(\gcd(c,p-1) = 1\)
		\item Wähle \(b\) zufällig
		\item \(s_{1} = g^{b} \cdot y^{c}\)
		\item \(s_{2} = -s_{1} \cdot (c^{-1}) \pmod{p-1}\)
		\item \(m = -s_{1} \cdot b \cdot (c^{-1}) \pmod{p-1}\)
	\end{enumerate}

	\subsubsection*{Angriff mit Bleichenbacher}
	Falls eine Signatur \((s_{1},s_{2})\) zur Nachricht \(m\) bekannt ist kann
	wie folgt eine Signatur gefälscht werden:
	\begin{enumerate}
		\item Berechne \(m^{-1} \bmod (p-1)\)
		\item Wähle eine zufällig Nachricht \(m'\)
		\item \(u = m' \cdot m^{-1} \pmod{p-1}\)
		\item \(v = s_{1} \cdot u \pmod{p-1}\)
		\item \(s_{1}' = \left(s_{1} \cdot (p-1)^{2}\right) + (v \cdot p) \bmod \left(p \cdot (p-1)\right)\)
		      \text{\footnotesize\textit{Chinesischer Restsatz}}
		\item \(s_{2}' = s_{2} \cdot u \pmod{p-1}\)
	\end{enumerate}
	Das Ergebnis: \((m',s_{1}',s_{2}')\), dabei ist zu beachten das \(m'\) nicht
	\(\bmod p\) genommen wird, d.h. es größer als \(p\). Ohne den ersten Schritt
	in der Verifizierung (\(1 \leq s_{1} \leq p-1\)) wäre diese Signatur gültig.

	\section{Hashfunktionen}
	Hashfunktionen sind Funktionen die von einer großen, potentiell
	unbeschränkten Menge in eine kleinere Menge abbilden, also:
	\[h_{k}: \{0,1\}^{*} \rightarrow \{0,1\}^{k}\]

	\subsection{Sicherheitseigenschaften}
	\subsubsection{Starke Kollisionsresistenz}
	Die Kollisionsresistenz \emph{(collision resistance)} bedeutet dass, es
	praktisch unmöglich ist zwei verschiedene Eingabewerte \(x\) und \(x'\) zu
	finden, die denselben Hashwert ergeben. \[x \neq x',\; h(x) = h(x')\]

	\subsubsection{Einwegeigenschaft}
	Die Einwegeigenschaft \emph{(pre-image resistance)} bedeutet dass, es
	praktisch unmöglich ist zu einem Hashwert \(y\) einen Eingabewert \(x\) zu
	finden, den die Hashfunktion auf \(y\) abbildet. \[h(x) = y\]

	\subsubsection{Schwache Kollisionsresistenz}
	Einen Zwischenschritt zwischen Kollisionsresistenz und Einwegeigenschaft
	stellt die schwache Kollisionsresistenz \emph{(second preimage resistance)}
	dar, dass bedeutet das es praktisch unmöglich ist zu einem gegebenen
	Eingabewert \(x\) einen davon verschieden Eingabewert \(x'\) zu finden, der
	den selben Hashwert ergibt. \[x \neq x',\; h(x) = h(x')\]

	\subsection{Angriffe}
	\subsubsection*{Kollisionsresistenz}
	\paragraph{Starke Kollisionsresistenz}
	Der Angreifer generriert eine zufällige Nachricht und berechnet für diese
	den Hashwert. Nach etwa \(n^{\frac{n}{1}}\) Nachrichten kann davon
	ausgegangen werden das der Angreifer eine Kollision findet. \par
	Aufgrund dessen ist die Hashfunktion \texttt{MD5} nicht mehr sicher, da
	diese eine Länge \(n = 128\) hat.
	\paragraph{Schwache Kollisionsresistenz}
	Der Angreifer generriert eine zufällig Nachricht und berechnet für diese
	den Hashwert und vergleicht diesen mit einem vorgegebenen Hashwert. Nach
	\(2^{n}\) Nachrichten kann davon ausgegangen werden das der Angreifer eine
	Nachricht findet welche den selben Hashwert besitzt. \par
	Diesbezüglich ist die Hashfunktion \texttt{MD5} immernoch sicher.

	\subsection{Modification Detection Code}
	\emph{Modification Detection Code} (MDC) ist ein Hashwert, welcher der
	Integritätsprüfung dient.
	\paragraph{Nachrichten Integritätsprüfung}
	\begin{itemize}[nolistsep]
		\item \underline{Asymetrisch:} Signatur
		\item \underline{Symmetrisch:} MAC
	\end{itemize}

	\subsection{Message Authentification Code}
	\emph{Message Authentication Codes} (MAC) sind ein symmetrisches Verfahren,
	um die Authentizität einer Nachricht sicherzustellen. Hierzu gibt es einen
	Signatur- und einen Verifikationsalgorithmus welche beide eine gemeinsames
	Geheimnis benötigen.
	\subsubsection{Hash-MAC}
	Um einen HMAC \(\sigma\) zu prüfen, erstellt der Empfänger selbst einen HMAC
	und prüft, ob diese identisch ist zu \(\sigma\). \par
	\(\textit{opad} = \text{0x5C} \quad \textit{ipad} = \text{0x36}\)
	\[\sig(k,\plaint) = h((k \oplus \textit{opad})|| h((k \oplus\textit{ipad})|| m))\]
	\subsubsection{CBC-MAC}
	Das Verfahren ist zu größen Teilen identisch mit dem normalen CBC-Verfahren,
	jeoch ist der Initialisierungsvektor fest. Der letzte Geheimtext ist der
	eigentliche MAC Code. \par
	Zu beachten ist jedoch das diese Verfahren zum einen \textbf{symmertisch}
	ist und zum anderen die \textbf{Länge für die Nachrichten festgesetzt} ist.

	\subsection{Gebrochene Algorithmen}
	\begin{center}
		\begin{tabular}{c| c }
			Algorithmus & Gebrochene Eigenschaft     \\ \hline
			MD5         & starke Kollisionsresistenz \\
			SHA-1       & starke Kollisionsresistenz
		\end{tabular}
	\end{center}

	\section{Protokolle}
	\subsection[Diffie-Hellman-Schlüsselaustausch]{Diffie-Hellman-\\Schlüsselaustausch}
	Alice und Bob haben einen gemeinsamen öffentlichen Schlüssel \((p,g)\),
	wobei \(p\) eine Primzahl ist und \(g\) eine Primitivwurzel von
	\(\mathbb{Z}_{p}\).
	\begin{enumerate}
		\item Alice wählt zufälliges $a \in [0;p-2]$ und berechnet $c = g^a \bmod p$
		      und übermittelt $c$ an Bob.
		\item Bob wählt zufälliges $b \in [0;p-2]$ und berechnet $d = g^b \bmod p$
		      und übermittelt $d$ an Alice.
		\item Alice berechnet nun $k = d^a \bmod p$
		\item Bob berechnet nun $k = c^b \bmod p$
	\end{enumerate}
	Wichtig bei der Angabe des Schlüssels muss evtl. mit \(\) gepaddet werden um
	die geforderte Schlüssellänge zu erreichen.

	\subsection{Station-to-Station-Protokoll}
	Eine Erweiterung des Diffie-Hellman-Schlüsselaustausch um einen
	Man-in-the-Middle-Angriff auszuschließen.
	\begin{enumerate}
		\item Alice wählt zufälliges $a \in [0;p-2]$ und berechnet $c = g^{a} \bmod p$
		      und übermittelt $c$ an Bob.
		\item Bob wählt zufälliges $b \in [0;p-2]$ und berechnet $k = g^{ab} \bmod p$
		\item Bob sendet nun $g^{b}$ sowie $z = \enc_{k}(s)$ mit $s = \sig_{\skey_{B}}(g^{a}||g^{b})$
		\item Alice $k = g^{ab} \bmod p$ und $s = \dec_{k}(z)$ sowie $\ver((g^{a}||g^{b}),s,\pkey_{B})$
		\item Alice sendet nun $z = \enc_{k}(s)$ mit $s = \sig_{\skey_{A}}(g^{b}||g^{a})$
		\item Bob entschlüsselt $s = \dec_{k}(z)$ und verifiziert $\ver((g^{b}||g^{a}),s,\pkey_{A})$
	\end{enumerate}
	Sollte die letzte Überprüfung korrekt sein, dann ist ein gemeinsamer Schlüssel gewählt.

	\subsection{SSl und TLS}
	TLS (\emph{Transport Socket Layer}) ist eine weiterentwicklung des von
	Netscape entwickelten Protokolls SSL (\emph{Secure Socket Layer}).
	Das Protokoll besteht aus 5 Teilprotokollen. TLS setzt auf die
	Transportschicht des OSI-Modell auf. Darurch kann TLS unabhängig von
	Anwendungen auf TCP aufgesetzt werden.  Den Teilprotokollen kommen dabei
	verschiedene Funktionen zu:
	\begin{itemize}
		\item Das \emph{TLS Handshake Protocol} initialisiert die
		      Verschlüsselung. Dieses Protokoll führt den Schlüsselaustausch
		      durch.
		\item Das \emph{TLS Change Cipher Spec Protocol} enthält bloß ein Byte
		      mit dem Inhalt \glqq 1\grqq. Dies dient dazu, die ausgehandelte
		      Verschlüsselung zu aktivieren.
		\item Das \emph{TLS Alert Protocol} meldet Fehler, die im Betrieb
		      aufgetreten sind.
		\item Das \emph{TLS Record Protocol} ist ein Dummy-Protokoll, dass die
		      Daten der Anwendungen weiterreicht.
		\item Das \emph{TLS Application Data Protocol} dient dazu, Daten mit den
		      ausgehandelten Konfigurationen zu verschlüsseln.
	\end{itemize}
	\subsubsection{TLS-Handshake}
	\begin{center}
		\documentclass[12pt]{standalone}
\usepackage{tikz}
\usetikzlibrary{calc,arrows,positioning}

\begin{document}

\begin{tikzpicture}
	\tikzstyle{block} = [text width=11cm]
	\coordinate (ClientTop) at (0,0);
	\coordinate[label=Client] (ClientBottom) at (0,8);
	\coordinate (ServerTop) at (5,0);
	\coordinate[label=Server] (ServerBottom) at (5,8);

	\coordinate[below=.3cm of ClientBottom] (A);
	\coordinate[below=.3cm of ServerBottom] (B);
	\coordinate[below=.8cm of B] (C);
	\coordinate[below=.8cm of A] (D);
	\coordinate[below=.8cm of C] (E);
	\coordinate[below=.8cm of D] (F);
	\coordinate[below=.8cm of E] (G);
	\coordinate[below=.8cm of F] (H);
	\coordinate[below=.8cm of G] (I);
	\coordinate[below=.8cm of H] (J);
	\coordinate[below=.8cm of I] (K);
	\coordinate[below=.8cm of J] (L);
	\coordinate[below=.8cm of K] (M);
	\coordinate[below=.8cm of L] (N);
	\coordinate[below=.8cm of N] (O);
	\coordinate[below=.8cm of M] (P);
	\coordinate[below=.8cm of O] (Q);
	\coordinate[below=.8cm of P] (R);
	\coordinate[below=.8cm of Q] (S);
	\coordinate[below=.8cm of R] (T);

	\draw[thick] (ClientTop)--(ClientBottom);
	\draw[thick] (ServerTop)--(ServerBottom);

	\draw[->,>=stealth,thick] (A) -- (B)  node[midway,sloped,above] {ClientHello};
	\draw[->,>=stealth,thick] (C) -- (D)  node[midway,sloped,above] {ServerHello};
	\draw[->,>=stealth,thick] (E) -- (F)  node[midway,sloped,above] {EncryptedExtensions};
	\draw[->,>=stealth,thick,dashed] (G) -- (H)  node[midway,sloped,above] {CertificateRequest};
	\draw[->,>=stealth,thick] (I) -- (J)  node[midway,sloped,above] {Certificate};
	\draw[->,>=stealth,thick] (K) -- (L)  node[midway,sloped,above] {CertificateVerify};
	\draw[->,>=stealth,thick] (M) -- (N)  node[midway,sloped,above] {Finished};
	\draw[->,>=stealth,thick,dashed] (O) -- (P)  node[midway,sloped,above] {Certificate};
	\draw[->,>=stealth,thick,dashed] (Q) -- (R)  node[midway,sloped,above] {CertificateVerify};
	\draw[->,>=stealth,thick] (S) -- (T)  node[midway,sloped,above] {Finished};
\end{tikzpicture}


\end{document}
		\footnotesize TLS-Handshake ab TLS 1.3
	\end{center}
	Die hervorgehobenen Nachrichten nur bei der Client-Authentifizierung.
	\begin{description}
		\item[ClientHello] Enthält die unterstützten:
		      \begin{itemize}[leftmargin=0pt]
			      \item TLS-Versionen
			      \item Cipher-Suites
			      \item Signaturverfahren
			      \item Elliptic Curve Diffie-Hellman (ECDH) Gruppen oder
			            Ephemeral Diffie-Hellman (EDH) Gruppen
			      \item Diffie-Hellman \emph{key shares}
			      \item Referenzen auf Pre-shared keys (PSK)
			      \item 32-Bit \texttt{client\_random}
		      \end{itemize}
		\item[ServerHello] enthält die ausgewählte:
		      \begin{itemize}[leftmargin=0pt]
			      \item TLS-Version
			      \item Cipher-Suite (Server wählt die stärkste)
			      \item ECDH Gruppe oder EDH Gruppe
			      \item Referenze auf PSK
			      \item 32-Bit \texttt{server\_random}
		      \end{itemize}
		\item[EncryptedExtensions] \;
		      \begin{itemize}[leftmargin=0pt]
			      \item \texttt{max\_fragment\_size}
			      \item \texttt{cookies}
			      \item \texttt{certificate\_authorities}
		      \end{itemize}
		\item[Certificate] Server (bzw. Client) schickt seinen öffentlichen
		      Schlüsselin einem Zertifikat, der Client (bzw. Server) prüft
		      das Zertifikat.
		\item[CertificateVerify] Der Server (bzw. Client) schickt seine
		      Unterschrift auf das Transkript der vorangegangenen
		      Handshake-Nachrichten. Die Challenge \texttt{client\_random} des
		      Clients (bzw. die challenge \texttt{server\_random} des Servers)
		      ist im Transkript enthalten und wird damit mit unterschrieben.
		\item[Finished] Sie enthalten jeweils einen mit dem gemeinsamen
		      Geheimnis gebildeten MAC über alle vorher beim handshake ausgetauschten
		      Nachrichten. Mit diesen MACs können Client und Server die Integrität und
		      den Absender aller Nachrichten im Handshake verifizieren.
	\end{description}
	\section{Primzahlen}
	2, 3, 5, 7, 11, 13, 17, 19, 23, 29, 31, 37, 41, 43, 47, 53, 59, 61, 67, 71,
	73, 79, 83, 89, 97, 101, 103, 107, 109, 113, 127, 131, 137, 139, 149, 151,
	157, 163, 167, 173, 179, 181, 191, 193, 197, 199, 211, 223, 227, 229, 233,
	239, 241, 251, 257, 263, 269, 271, 277, 281, 283, 293, 307, 311, 313, 317,
	331, 337, 347, 349, 353, 359, 367, 373, 379, 383, 389, 397, 401, 409, 419,
	421, 431, 433, 439, 443, 449, 457, 461, 463, 467, 479, 487, 491, 499, 503,
	509, 521, 523, 541, 547, 557, 563, 569, 571, 577, 587, 593, 599, 601, 607,
	613, 617, 619, 631, 641, 643, 647, 653, 659, 661, 673, 677, 683, 691, 701,
	709, 719, 727, 733, 739, 743, 751, 757, 761, 769, 773, 787, 797, 809, 811,
	821, 823, 827, 829, 839, 853, 857, 859, 863, 877, 881, 883, 887, 907, 911,
	919, 929, 937, 941, 947, 953, 967, 971, 977, 983, 991, 997, 1009, 1013,
	1019, 1021, 1031, 1033, 1039, 1049, 1051, 1061, 1063, 1069, 1087, 1091,
	1093, 1097, 1103, 1109, 1117, 1123, 1129, 1151, 1153, 1163, 1171, 1181,
	1187, 1193, 1201, 1213, 1217, 1223, 1229, 1231, 1237, 1249, 1259
\end{multicols*}
\end{document}
