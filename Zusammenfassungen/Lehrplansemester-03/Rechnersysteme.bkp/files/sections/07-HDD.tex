\section{HDD}

\subsection{Begriffe}
\begin{description}
\item[Spur]
Ein Kreis auf der Festplatte auf dem die Sektoren nacheinander liegen. Die
Spuren sind konzentrisch auf der Festplatte gelegt.

\item[Sektor]
Ein Stück aus einer Spur mit fester Breite. Die Sektorlänge ist z.B. in Bytes
($512$ oder $4096$ Bytes) oder Mikrometer angegeben. Jeder Sektor enthält zudem
noch $40$ bis $100$ Bytes für Verwaltungsinformationen (\emph{Präambel}) und 
Fehlererkennung/-korrektur.

\item[Zylinder]
Wenn die Festplatte aus mehreren übereinander liegenden Scheiben besteht,
besteht ein Zylinder allen Spuren die genau übereinander auf den Scheiben
liegen. Wenn es z.B. 2 Scheiben gibt, dann sind alle $2$ Spuren die ganz außen
auf jeder Scheibe liegen ein Zylinder; genauso machen alle Spuren Nr. $16$ auf
jeder Scheibe zusammen ein Zylinder aus: Zylinder $16$. Ein Zylinder ist also
eine Erweiterung der Spur auf vertikalen Ebene wenn es mehrere Scheiben gibt. Es
folgt, auf einem Zylinder der aus $n$ übereinander liegenden Spuren besteht und
wenn jede dieser Spuren $k$ Bytes hat, dann hat ein Zylinder $n \cdot k$ Bytes.
Folgt auch, dass eine Angabe wie $32$ Zylinder bedeutet dass jede Scheibe $32$
Spuren hat.
\end{description}

\subsection{Datenübertragungszeit}
Die mittlere Datenübertragungszeit $T_{\text{average}}$ einer
Festplatte hinsichtlich der auf einer Spur befindlichen Daten kann
vereinfacht mittels der folgenden Funktion dargestellt werden:
\[
    T_{\text{average}} := T_{s} + T_{r} + T_{t}
\]
Die mittlere Positionierungszeit $T_{s}$ setzt sich dabei zusammen aus der 
Positionierungszeit $MT_{\text{sr}}$ des Schreib/Lese-Kopfs in die
richtige radiale Position zum Lesen/Schreiben und der Zeit $MT_{\text{sc}}$
zum wechseln auf benachbarte Spuren $S_{c}$, daraus leiten sich folgenden Funktionen ab:
\[
 T_{s} := T_{\text{sr}} + T_{\text{sc}}
\]
\[
 T_{\text{sr}} := MT_{\text{sr}}\cdot (S-S_{c})
\]
\[
 T_{\text{sc}} := MT_{\text{sc}}\cdot S_{c}
\]
Sollten jedoch alle Spuren $S$ bis auf die erste benachbart sein, dann gilt für
die Positionierungszeit folgenden Funktion:
\[
 T_{s} := MT_{\text{sr}} + MT_{\text{sc}}\cdot (S-1)
\]
Der Drehverzug $T_{r}$ fällt bei jedem Spurwechsel an und ist die Verzögerung,
welche anfällt bis sich der gesuchte Sektor auf der rotierenden Festplatte unter
dem Schreib/Lese-Kopf befindet. Die Drehverzugszeit ist dabei abhängig von der
Rotationsgeschwindigkeit $v$ welche in \emph{RPM} angegeben wird.
\[
 T_{r} := \left(\cfrac{60\cdot 1000}{2v}\right)\cdot S
\]
Die Übertragungszeit $T_{t}$ ist die eigentliche Zeit welche benötigt wird und die Daten
zu übertragen; diese ist abhängig von der Rotationsgeschwindigkeit $v$ sowie der Anzahl
der genutzten Sektoren $N_{u}$ pro Spur und der Gesamtanzahl der Sektoren $N_{t}$ pro Spur.
\[
 T_{t} := \left(\cfrac{60\cdot 1000}{v}\right)\cdot \cfrac{N_{u}}{N_{t}}\cdot S
\]
Die Anzahl der benötigten Spuren lässt wie folgt berechnen:
\[
\text{Spuren} = \cfrac{\text{Datei}_\text{Gr}}{\text{Oberflächen}_\text{Anz}\cdot \text{Sektoren}_\text{Anz}\cdot \text{Sektor}_\text{Gr}}
\]

\subsection{Speicherdirektzugriff}
Der Speicherdirektzugriff ermöglicht es angeschlossenen Peripheriegeräten ohne Umweg über
die CPU mit dem Arbeitsspeicher zu kommuniziren. Ein Vorteil des Speicherdirektzugriffs
ist die schneller Datenübertargung bei einer gleichzeitigen Entlastung des Prozessors.
Der DMA-Controller muss jedoch die Daten zwangsläufig über das gleiche Bussystem übertragen
wie die CPU. \par
Die Buszyklen $C$ welche für einen $n$-Bit-DMA-Transfer benötigten werden kann
mittels folgender Funktion, in Abhängigkeit von der Umdrehungsdauer $T_{r}$ (in ms), 
der Sektorgröße $S$ und der Sektorenanzahl $A$,ermittelt werden:
\[
 C = \left\lceil\cfrac{A\cdot S\cdot 8}{n\cdot T_{r}}\right\rceil
\]
